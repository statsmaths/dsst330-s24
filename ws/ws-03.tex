\documentclass{tufte-handout}

\usepackage{amssymb,amsmath}
% \usepackage{mathspec}
\usepackage{graphicx,grffile}
\usepackage{longtable}
\usepackage{booktabs}
\usepackage{mathtools}

\newtheorem{mydef}{Definition}
\newtheorem{thm}{Theorem}

\DeclareMathOperator*{\argmin}{arg\,min}
\DeclareMathOperator*{\argmax}{arg\,max}
\newcommand{\Lim}[1]{\raisebox{0.5ex}{\scalebox{0.8}{$\displaystyle \lim_{#1}\;$}}}
\newcommand{\E}{\mathbb{E}}
\newcommand{\Prob}{\mathbb{P}}
\newcommand{\V}{\text{Var}}
\newcommand{\iid}{\stackrel{iid}{\sim}}
\newcommand{\cblack}{\color{Black}}
\newcommand{\cblue}{\color{MidnightBlue}}

\providecommand{\tightlist}{%
  \setlength{\itemsep}{0pt}\setlength{\parskip}{0pt}}

\setlength{\parindent}{0em}
\setlength{\parskip}{12pt}

\begin{document}

\justify

{\LARGE Worksheet 03}

\vspace*{18pt}


\textbf{1}. Let $X_1, \ldots, X_n \sim N(\mu_X, \sigma_X^2)$ be a random sample. Using
the results from the handout, construct a pivot statistic $T$ as a function
of $\bar{X}$, $S_X^2$, $\mu_X$, and $\sigma_X^2$ that has a distribution of
$t(n-1)$. Do not simplify.

\textbf{2}. Simplify the form of the $T$ statistic. It should no longer have any
$\sigma_X^2$ terms (in fact this is the whole point of this specific form). 
Try to write the solution with $(\mu - \bar{X})$ in the numerator and
everything else in the denominator.

\textbf{3}. Let $t_\alpha(k)$ be the tail probability of a T-distribution with $k$
degrees of freedom, just as we had with $z_\alpha$ on the handout. Following
the same procedure with the example on the handout, construct a confidence
interval with confidence level $(1 - \alpha)$ for $\mu_X$. Write the solution
as $\bar{X} \pm \Delta$ for some $\Delta$.

\textbf{4}. We will go back to the story about the fish. Say we sample $25$ fish and
have a sample mean of $12.1$ centimeters and a sample variance of $6$
centimeters squared. Given that $t_{0.01/2}(24)$ is approximately equal
to $2.797$, derive the confidence interval for the mean.

% qt(1 - 0.01/2, df = 24)
% 2.797 * sqrt(6/24)

\textbf{5}. Let $C_k \sim \chi^2(k)$ for every integer $k$. Use Chebychev's Inequality to show
that for any $\epsilon > 0$, we have:
\begin{align*}
\lim_{k \rightarrow \infty} \mathbb{P} \left[ |C_k/k - 1| \geq \epsilon \right] = 0
\end{align*}
In this case we say that $C_k$ limits in probability to 1, written as
$C_k \rightarrow_P 1$.

\textbf{6}. Let $Y_n \rightarrow_P y$ for a constant $y$, $f$ is a real-valued function that
is invertable around the neighborhood of $y$, and $X$ is another random variable.
Then, Slutsky's Theorem says that $g(Y_n) \cdot X$ limits in probability to
$g(y) \cdot X$. Use this to show that the $T$ distribution limits to the standard
normal as the degrees of freedom limit to infinity.


\end{document}
