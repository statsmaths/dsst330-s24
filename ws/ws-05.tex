\documentclass{tufte-handout}

\usepackage{amssymb,amsmath}
% \usepackage{mathspec}
\usepackage{graphicx,grffile}
\usepackage{longtable}
\usepackage{booktabs}
\usepackage{mathtools}

\newtheorem{mydef}{Definition}
\newtheorem{thm}{Theorem}

\DeclareMathOperator*{\argmin}{arg\,min}
\DeclareMathOperator*{\argmax}{arg\,max}
\newcommand{\Lim}[1]{\raisebox{0.5ex}{\scalebox{0.8}{$\displaystyle \lim_{#1}\;$}}}
\newcommand{\E}{\mathbb{E}}
\newcommand{\Prob}{\mathbb{P}}
\newcommand{\V}{\text{Var}}
\newcommand{\iid}{\stackrel{iid}{\sim}}
\newcommand{\cblack}{\color{Black}}
\newcommand{\cblue}{\color{MidnightBlue}}

\providecommand{\tightlist}{%
  \setlength{\itemsep}{0pt}\setlength{\parskip}{0pt}}

\setlength{\parindent}{0em}
\setlength{\parskip}{12pt}

\begin{document}

\justify

{\LARGE Worksheet 05}

\vspace*{18pt}


\textbf{1}. We will start with the easier, though much less commonly used, task of
comparing the variances between the samples before moving onto the confidence
interval of the mean. Using what we know about the distributions of $S_X^2$ and
$S_Y^2$, build a pivot based on the scaled ratio of these two quantities that
has an F-distribution.

\textbf{2}. Rearrange your previous result to get a confidence interval for the ratio
$\sigma_Y^2 / \sigma_X^2$. Note that the F distribution is not symmetric.

\textbf{3}. What is $\mathbb{E}[\bar{X} - \bar{Y}]$? Make use of the properties that we
already know to make this relatively easy. You should see that this is an
unbiased estimator of the difference in the means.

\textbf{4}. What is $\text{Var}[\bar{X} - \bar{Y}]$? Make use of the properties that we
already know to make this relatively easy. The result should imply that the
difference is a consistent estimator of the difference in sample means.

\textbf{5}. If $\mathcal{G}_X$ and $\mathcal{G}_Y$ are both normally distributed, then
$\bar{X} - \bar{Y}$ also has a normal distribution. As we did in the one-sample
case, construct a pivot $Z$ that scales this difference to have a standard 
normal distribution.

\textbf{6}. Assume that $\sigma_X^2 = \sigma_Y^2$, which we can write as just $\sigma^2$.
Take the definition of the pooled sample variance and multiply both sides by
$(n + m - 2)$ and divide by $\sigma^2$. If we assume that $\mathcal{G}_X$ and
$\mathcal{G}_Y$ are both normally distributed, show that $S_p^2$ is a scaled
version of a chi-squared distribution. What are its degrees of freedom?

\textbf{7}. Put together the previous results to generate a pivot $T$ that has a T-distribution.
Rearrange the terms to get a confidence interval for the difference in means.


\end{document}
