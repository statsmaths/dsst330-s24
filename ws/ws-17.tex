\documentclass{tufte-handout}

\usepackage{amssymb,amsmath}
% \usepackage{mathspec}
\usepackage{graphicx,grffile}
\usepackage{longtable}
\usepackage{booktabs}
\usepackage{mathtools}

\newtheorem{mydef}{Definition}
\newtheorem{thm}{Theorem}

\DeclareMathOperator*{\argmin}{arg\,min}
\DeclareMathOperator*{\argmax}{arg\,max}
\newcommand{\Lim}[1]{\raisebox{0.5ex}{\scalebox{0.8}{$\displaystyle \lim_{#1}\;$}}}
\newcommand{\E}{\mathbb{E}}
\newcommand{\Prob}{\mathbb{P}}
\newcommand{\V}{\text{Var}}
\newcommand{\iid}{\stackrel{iid}{\sim}}
\newcommand{\cblack}{\color{Black}}
\newcommand{\cblue}{\color{MidnightBlue}}

\providecommand{\tightlist}{%
  \setlength{\itemsep}{0pt}\setlength{\parskip}{0pt}}

\setlength{\parindent}{0em}
\setlength{\parskip}{12pt}

\begin{document}

\justify

{\LARGE Worksheet 17}

\vspace*{18pt}


\textbf{1}. Consider a prior distribution $p \sim Beta(\alpha, \beta)$ for some fixed
$\alpha$ and $\beta$ for a likelihood given by $X|p \sim Bin(n, p)$. Derive
the posterior distribution $p|X$.

\textbf{2}. For reasons that we will explore in more next time, the Bayesian point estimator
(the best single-number estimator) is the expected value of the posterior
distribution. Under the set up from the previous question, what the Bayesian
point estimator $\widehat{p}$ in terms of $X$, $\alpha$, and $\beta$?

\textbf{3}. Consider observing $X \sim Bin(n, p)$. We know that the MLE estimator of $p$ is
given by $\hat{p}_{MLE} = X / n$. The Binomial comes from doing $n$ Bernoulli
trials and adding the number of 1s. Consider creating a new $Y$ in which we 
artificially augment the data $X$ by adding (in effect) an extra 0 and an extra
1. In other words, we create a $Y = X + 1$ with the assumption that
$Y \sim Bin(n+2, p)$. What is the MLE of $p$ using the data from the augmented
data $Y$? Where have you seen this before?

\textbf{4}. Consider your solution to the previous set of questions. If $\alpha$ and $\beta$
are non-negative integers, how could you describe the Bayesian estimator based
on adding data to $X$? 

\textbf{5}. The standard uniform distribution is equivalent to $Beta(1,1)$. In the notes
I started by implicitly assuming that this was a fairly neutral starting
assumption for indicating that we do not have any strong prior knowledge of
$p$. Based on your results above, what would actually seem to be the best
netural position if we do not want the prior to have a strong influence on 
the posterior mean?


\end{document}
