\documentclass{tufte-handout}

\usepackage{amssymb,amsmath}
% \usepackage{mathspec}
\usepackage{graphicx,grffile}
\usepackage{longtable}
\usepackage{booktabs}
\usepackage{mathtools}

\newtheorem{mydef}{Definition}
\newtheorem{thm}{Theorem}

\DeclareMathOperator*{\argmin}{arg\,min}
\DeclareMathOperator*{\argmax}{arg\,max}
\newcommand{\Lim}[1]{\raisebox{0.5ex}{\scalebox{0.8}{$\displaystyle \lim_{#1}\;$}}}
\newcommand{\E}{\mathbb{E}}
\newcommand{\Prob}{\mathbb{P}}
\newcommand{\V}{\text{Var}}
\newcommand{\iid}{\stackrel{iid}{\sim}}
\newcommand{\cblack}{\color{Black}}
\newcommand{\cblue}{\color{MidnightBlue}}

\providecommand{\tightlist}{%
  \setlength{\itemsep}{0pt}\setlength{\parskip}{0pt}}

\setlength{\parindent}{0em}
\setlength{\parskip}{12pt}

\begin{document}

\justify

{\LARGE Worksheet 19}

\vspace*{18pt}


\textbf{1}. Let $X \sim N(\mu, \sigma^2)$, with $\sigma^2 > 0$ a fixed and known constant.
(a) Compute the Fisher Information $\mathcal{I}(\mu)$. (b) The MLE for $\mu$
is equal to $X$ (generally it's the mean, but in the one-observation case the
mean is equal to $X$). Find the efficency of the MLE.

\textbf{2}. Let $X \sim Poisson(\lambda)$. (a) Compute the Fisher Information $\mathcal{I}(\lambda)$.
(b) The MLE for $\lambda$ is equal to $X$ (generally it's the mean, but in the one-observation
case the mean is equal to $X$). Find the efficency of the MLE.

\textbf{3}. Let $X \sim Binomial(n, p)$ with $n>0$ a fixed and known constant.
(a) Compute the Fisher Information $\mathcal{I}(p)$.\footnote{
  Try to simplify this as much as possible. You should be able to
  get something that has a denominator equal to $p(1-p)$.
} (b) The MLE for $p$ is
equal to $X/n$. Find the efficency of the MLE.


\end{document}
