\documentclass{tufte-handout}

\usepackage{amssymb,amsmath}
% \usepackage{mathspec}
\usepackage{graphicx,grffile}
\usepackage{longtable}
\usepackage{booktabs}
\usepackage{mathtools}

\newtheorem{mydef}{Definition}
\newtheorem{thm}{Theorem}

\newcommand{\Lim}[1]{\raisebox{0.5ex}{\scalebox{0.8}{$\displaystyle \lim_{#1}\;$}}}
\newcommand{\iid}{\overset{\mathrm{i.i.d.}}{\sim}}

\providecommand{\tightlist}{%
  \setlength{\itemsep}{0pt}\setlength{\parskip}{0pt}}

\setlength{\parindent}{0em}
\setlength{\parskip}{12pt}

\begin{document}

\justify

{\LARGE Worksheet 01}

\vspace*{18pt}


\textbf{1}. We will give more formal definitions later, but for now
define a probability of an event to be a number between 0 and 1 that
indicates how likely an event would be to happen. For example, a
value of 0 indicates that it will never happen, a value of 1 that it
will always happen. This matches the way that the word `probability' is
colloquial used in a non-technical context. While in casual conversatoin
most people refer to the number as a percentage or fraction, it will
be good to start thinking of them as decimals. Given this, give approximate
values for the probability of the following events:

\begin{itemize}[\label={}]
  \tightlist
  \item {(a)} A randomly selected M\&M will be blue.
  \item {(b)} A randomly selected car in Virginia is electric.
  \item {(c)} A randomly selected book starts with the word `The'.
  \item {(d)} An NBA basketball player will make a free throw.
  \item {(e)} A pregnancy results in having twins.
  \item {(f)} A clover will be a four-leaf clover.
  \item {(g)} A letter will be lost by the U.S. postal service.
  \item {(h)} Someone born in the U.S. in the year 2000 is named Taylor.
\end{itemize}


\end{document}
