\documentclass{tufte-handout}

\usepackage{amssymb,amsmath}
% \usepackage{mathspec}
\usepackage{graphicx,grffile}
\usepackage{longtable}
\usepackage{booktabs}
\usepackage{mathtools}

\newtheorem{mydef}{Definition}
\newtheorem{thm}{Theorem}

\DeclareMathOperator*{\argmin}{arg\,min}
\DeclareMathOperator*{\argmax}{arg\,max}
\newcommand{\Lim}[1]{\raisebox{0.5ex}{\scalebox{0.8}{$\displaystyle \lim_{#1}\;$}}}
\newcommand{\E}{\mathbb{E}}
\newcommand{\Prob}{\mathbb{P}}
\newcommand{\V}{\text{Var}}
\newcommand{\iid}{\stackrel{iid}{\sim}}
\newcommand{\cblack}{\color{Black}}
\newcommand{\cblue}{\color{MidnightBlue}}

\providecommand{\tightlist}{%
  \setlength{\itemsep}{0pt}\setlength{\parskip}{0pt}}

\setlength{\parindent}{0em}
\setlength{\parskip}{12pt}

\begin{document}

\justify

{\LARGE Worksheet 07}

\vspace*{18pt}


\textbf{1}. Consider a sample of size $6$ with the following values: $0$, $1$, $5$, $7$, $12$.
What are the sample mean and sample variance $\bar{X}$ and $S^2_X$?

% x <- c(0, 1, 5, 7, 12)
% var(x)

\textbf{2}. Consider collecting data from two populations. We collect $n=25$ observations
from the first group, with sample mean $7$ and sample variance $9$. From the 
second group, we have $m=30$ samples and a mean of $4$ with a sample variance
of $4$. What is the pooled variance $S^2_p$?

\textbf{3}. Using the data from above, construct a $99$\% confidence interval for the 
difference in the means. You can use the fact that $t_{0.99/2}(54) = 2.67$. 

% qt(1 - 0.01/2, df = 54)

\textbf{4}. Using the data from above, run a hypothesis test to see if the samples have the
same variance with a $99$\% confidence level. Use the fact that
$f_{0.99/2}(24, 29) = 2.76$ and $f_{1-0.99/2}(24, 29) = 0.347$. 

% qf(1 - 0.01/2, df1 = 24, df2 = 29)
% qf(0.01/2, df1 = 24, df2 = 29)


\end{document}
