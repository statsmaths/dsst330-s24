\documentclass{tufte-handout}

\usepackage{amssymb,amsmath}
% \usepackage{mathspec}
\usepackage{graphicx,grffile}
\usepackage{longtable}
\usepackage{booktabs}
\usepackage{mathtools}

\newtheorem{mydef}{Definition}
\newtheorem{thm}{Theorem}

\DeclareMathOperator*{\argmin}{arg\,min}
\DeclareMathOperator*{\argmax}{arg\,max}
\newcommand{\Lim}[1]{\raisebox{0.5ex}{\scalebox{0.8}{$\displaystyle \lim_{#1}\;$}}}
\newcommand{\E}{\mathbb{E}}
\newcommand{\Prob}{\mathbb{P}}
\newcommand{\V}{\text{Var}}
\newcommand{\iid}{\stackrel{iid}{\sim}}
\newcommand{\cblack}{\color{Black}}
\newcommand{\cblue}{\color{MidnightBlue}}

\providecommand{\tightlist}{%
  \setlength{\itemsep}{0pt}\setlength{\parskip}{0pt}}

\setlength{\parindent}{0em}
\setlength{\parskip}{12pt}

\begin{document}

\justify

{\LARGE Worksheet 07}

\vspace*{18pt}


\textbf{1}. Our goal is, broadly, to produce two independent random variabels that will have
chi-squared distributions under the null hypothesis. One will depend only on
the sample means and the other only on the sample variances. The latter is the
easier of the two, so let's start there. Consider the following quantity:
\begin{align*}
\sum_{j=1}^k \frac{(n_j - 1) S_j^2}{\sigma^2} &= \frac{(n_1 - 1) S_1^2}{\sigma^2} + \cdots + \frac{(n_j - 1) S_j^2}{\sigma^2}
\end{align*}
What is its distribution? Note that this result does not depend on the null
hypothesis

\textbf{2}. Now, let's work out another chi-squared distribution based only on the means.
What is the distribution of the following quantity, where $\mu$ is the
hypothesized mean of all the blocks?
\begin{align*}
\left[\frac{\bar{X}_j - \mu}{\sqrt{\sigma^2 / n_j}}\right]^2
\end{align*} 

\textbf{3}. Next, what is this quantity?
\begin{align*}
\left[\frac{\bar{X} - \mu}{\sqrt{\sigma^2 / N}}\right]^2
\end{align*} 

\textbf{4}. Using a similar derivation from worksheet 2, we can show that the following
is true:
\begin{align*}
\sum_{j=1}^K \left[\frac{\bar{X}_j - \mu}{\sqrt{\sigma^2 / n_j}}\right]^2 &=
\sum_{j=1}^K \left[\frac{\bar{X}_j - \bar{X}}{\sqrt{\sigma^2 / n_j}}\right]^2 +
\left[\frac{\bar{X} - \mu}{\sqrt{\sigma^2 / N}}\right]^2
\end{align*} 
Based on this and your previous results, what is the distribution of the 
second sum in the equation above?

\textbf{5}. Put all of the previous results together to construct an F-statistic for the
hypothesis. Simplify the cancel the unknown value $\sigma^2$. While technically
either way is valid, put the chi-squared based on the means in the numerator and
the chi-squared based on the variances in the denominator. This will follow the
typical convention. Make sure to write down the distribution of the test statistic
under $H_0$.


\end{document}
