\documentclass{tufte-handout}

\usepackage{amssymb,amsmath}
% \usepackage{mathspec}
\usepackage{graphicx,grffile}
\usepackage{longtable}
\usepackage{booktabs}
\usepackage{mathtools}

\newtheorem{mydef}{Definition}
\newtheorem{thm}{Theorem}

\DeclareMathOperator*{\argmin}{arg\,min}
\DeclareMathOperator*{\argmax}{arg\,max}
\newcommand{\Lim}[1]{\raisebox{0.5ex}{\scalebox{0.8}{$\displaystyle \lim_{#1}\;$}}}
\newcommand{\E}{\mathbb{E}}
\newcommand{\Prob}{\mathbb{P}}
\newcommand{\V}{\text{Var}}
\newcommand{\iid}{\stackrel{iid}{\sim}}
\newcommand{\cblack}{\color{Black}}
\newcommand{\cblue}{\color{MidnightBlue}}

\providecommand{\tightlist}{%
  \setlength{\itemsep}{0pt}\setlength{\parskip}{0pt}}

\setlength{\parindent}{0em}
\setlength{\parskip}{12pt}

\begin{document}

\justify

{\LARGE Worksheet 15}

\vspace*{18pt}


\textbf{1}. Consider a simple linear regression where the first $n/2$ values of $x_i$ are 
zero and the second $n/2$ values of $x_i$ are 1. We can use this to model the
mean of a variable that depends on whether the corresponding $x_i$ is in group
0 or group 1. Specifically, how do the means of $Y_i$ in these two groups
correspond to the parameters $b_0$ and $b_1$?

\textbf{2}. Let $\bar{y}_A$ be the mean of the first $n/2$ values of $Y_i$ and let $\bar{y}_B$
be the mean of the second $n/2$ values of $Y_i$. Consider the following form of the
MLE for $\widehat{b}_1$:\footnote{
  It can be shown that it is equivalent to the form on Worksheet 14.
}
\begin{align*}
\widehat{b}_1 &= \frac{\sum_i (y_i - \bar{y}) (x_i)}{\sum_i (x_i - \bar{x})^2}.
\end{align*}
Find a simple formula for $\widehat{b}_1$ in terms of $\bar{y}_A$ and $\bar{y}_B$.

\textbf{3}. Continuing from the previous question, find a simple formula for $\widehat{b}_0$
in terms of $\bar{y}_A$ and $\bar{y}_B$.

\textbf{4}. What is the connection between the linear regression here and a two-sample T-test
for the means across the two groups?


\end{document}
