\documentclass{tufte-handout}

\usepackage{amssymb,amsmath}
% \usepackage{mathspec}
\usepackage{graphicx,grffile}
\usepackage{longtable}
\usepackage{booktabs}
\usepackage{mathtools}

\newtheorem{mydef}{Definition}
\newtheorem{thm}{Theorem}

\DeclareMathOperator*{\argmin}{arg\,min}
\DeclareMathOperator*{\argmax}{arg\,max}
\newcommand{\Lim}[1]{\raisebox{0.5ex}{\scalebox{0.8}{$\displaystyle \lim_{#1}\;$}}}
\newcommand{\E}{\mathbb{E}}
\newcommand{\Prob}{\mathbb{P}}
\newcommand{\V}{\text{Var}}
\newcommand{\iid}{\stackrel{iid}{\sim}}
\newcommand{\cblack}{\color{Black}}
\newcommand{\cblue}{\color{MidnightBlue}}

\providecommand{\tightlist}{%
  \setlength{\itemsep}{0pt}\setlength{\parskip}{0pt}}

\setlength{\parindent}{0em}
\setlength{\parskip}{12pt}

\begin{document}

\justify

{\LARGE Worksheet 20}

\vspace*{18pt}


\textbf{1}. Find the Jeffreys prior for estimating the estimating the mean of a normal
distribution with a known variance $\sigma^2$. You can assume we have only one
observation $X$. What is the corresponding
Bayesian point estimator and how does it compare to the MLE?

\textbf{2}. Find the Jeffreys prior for estimating the estimating the parameter $p$ from a
Binomial with a known value $n$. What is the corresponding Bayesian point estimator?
What does this mean in the case when $n=1$ and $X=0$ and in the case when $n=1$ and $X=1$?

\textbf{3}. Find the Jeffreys prior for estimating the estimating the parameter $\lambda$ from
a Poisson. Write down a formula that gives, up to a constant,
the posterior distribution. Note that you will not be able to relate this to a 
known distribution on our chart.

\textbf{4}. The Fisher information for the geometric distribution is $\mathcal{I}(p) = \frac{(1-p)}{p^2}$.
Find the Jeffreys prior for estimating the estimating the parameter $p$ from
a geometric distribution. What is, more-or-less, this distribution?\footnote{
  It should line up with one of the results on the table, but the hyperparameter
  is out of bounds. That's okay though. It just means we have an improper prior.
  All of the results still hold. 
} What is the corresponding Bayesian point estimator? Using previous results,
you should be able to do this for a sample of size $n$.


\end{document}
