\documentclass{tufte-handout}

\usepackage{amssymb,amsmath}
% \usepackage{mathspec}
\usepackage{graphicx,grffile}
\usepackage{longtable}
\usepackage{booktabs}
\usepackage{mathtools}

\newtheorem{mydef}{Definition}
\newtheorem{thm}{Theorem}

\DeclareMathOperator*{\argmin}{arg\,min}
\DeclareMathOperator*{\argmax}{arg\,max}
\newcommand{\Lim}[1]{\raisebox{0.5ex}{\scalebox{0.8}{$\displaystyle \lim_{#1}\;$}}}
\newcommand{\E}{\mathbb{E}}
\newcommand{\Prob}{\mathbb{P}}
\newcommand{\V}{\text{Var}}
\newcommand{\iid}{\stackrel{iid}{\sim}}
\newcommand{\cblack}{\color{Black}}
\newcommand{\cblue}{\color{MidnightBlue}}

\providecommand{\tightlist}{%
  \setlength{\itemsep}{0pt}\setlength{\parskip}{0pt}}

\setlength{\parindent}{0em}
\setlength{\parskip}{12pt}

\begin{document}

\justify

{\LARGE Worksheet 09}

\vspace*{18pt}


\textbf{1}. Find the MLE estimator for the estimation of the parameter $\lambda$ from
i.i.d. observations of an exponentialy distributed random variable.

\textbf{2}. Find the MLE estimator for the estimation of the variance from
i.i.d. observations of an exponentialy distributed random variable.
Hint: This is easily derived from the previous result. Should not 
require any new derivatives.

\textbf{3}. Find the MLE estimator for the estimation of the parameter $p$ from
i.i.d. observations of a Bernoulli distributed random variable. Hint:
When you set the derivative equal to zero, multiple by $\frac{1}{n}$ to
write the equation in terms of just $\bar{X}$ and $\hat{p}$. 

\textbf{4}. Find the MLE estimator for the estimation of the parameters $\mu$ and $\sigma^2$
from i.i.d. observations of a normally distributed random variable. Hint: We
want to think of $\sigma^2$ as a single parameter (not the square of a parameter).
I recommend using $v = \sigma^2$ to keep this clear. Also, find $\hat{\mu}$ first.
You can find the MLE for the mean without knowing the MLE of the variance.

\textbf{5}. What is the bias of the MLE estimator for the variance from a normal distribution
with unknown mean and variance? Hint: Use what we know about $S_X^2$ to make this
relatively easy.


\end{document}
