\documentclass{tufte-handout}

\usepackage{amssymb,amsmath}
% \usepackage{mathspec}
\usepackage{graphicx,grffile}
\usepackage{longtable}
\usepackage{booktabs}
\usepackage{mathtools}

\newtheorem{mydef}{Definition}
\newtheorem{thm}{Theorem}

\DeclareMathOperator*{\argmin}{arg\,min}
\DeclareMathOperator*{\argmax}{arg\,max}
\newcommand{\Lim}[1]{\raisebox{0.5ex}{\scalebox{0.8}{$\displaystyle \lim_{#1}\;$}}}
\newcommand{\E}{\mathbb{E}}
\newcommand{\Prob}{\mathbb{P}}
\newcommand{\V}{\text{Var}}
\newcommand{\iid}{\stackrel{iid}{\sim}}
\newcommand{\cblack}{\color{Black}}
\newcommand{\cblue}{\color{MidnightBlue}}

\providecommand{\tightlist}{%
  \setlength{\itemsep}{0pt}\setlength{\parskip}{0pt}}

\setlength{\parindent}{0em}
\setlength{\parskip}{12pt}

\begin{document}

\justify

{\LARGE Worksheet 10}

\vspace*{18pt}


\textbf{1}. \textbf{(Ratio Test)} Let $X_1, \ldots, X_n \iid Exp(\lambda)$.
What is the test statistic $\Lambda$ for the corresponding likelihood 
ratio test for the null hypothesis $H_0: \lambda = 1$.

\textbf{2}. \textbf{(Ratio Test)} Let $X_1, \ldots, X_n \iid Poisson(\lambda)$.
What is the test statistic $\Lambda$ for the corresponding likelihood 
ratio test for the null hypothesis $H_0: \lambda = 1$.

\textbf{3}. \textbf{(Ratio Test)} Let $X_1, \ldots, X_n \iid Bernoulli(p)$.
What is the test statistic $\Lambda$ for the corresponding likelihood 
ratio test for the null hypothesis $H_0: p = 0.2$.

\textbf{4}. \textbf{(Ratio Test)} Let $X \sim Bin(n, p_1)$ and $Y \sim Bin(n, p_2)$ be
independent random variables, assuming that $n$ is a known quantity. We want
to test the hypothesis that $H_0: p_1 = p_2$. What are the corresponding
$\Theta$ and $\Theta_0$ in
our updated formulation of hypothesis testing?\footnote{
  We will derive the actual test itself in a more general form next
  class.
} If we use a Likelihood Ratio Test for this hypothesis, how many degrees of
freedom should $\Lambda$ have?

\textbf{5}. \textbf{(Ratio Test)} Recall that we used the one-sample ANOVA test with the
null-hypothesis that the means of $K$ samples are all the same. Write down and
describe the values of $\Theta$ and $\Theta_0$ that correspond to this test. 
If we use a Likelihood Ratio Test for this hypothesis, how many degrees of
freedom should $\Lambda$ have?

\textbf{6}. \textbf{(MLE Practice)} Let $X_1, \ldots, X_n \iid Uniform(0, a)$. Find the MLE
estimator for $a$. Note: You cannot do this using the derivative. Just think about it!


\end{document}
