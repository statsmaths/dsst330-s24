\documentclass{tufte-handout}

\usepackage{amssymb,amsmath}
% \usepackage{mathspec}
\usepackage{graphicx,grffile}
\usepackage{longtable}
\usepackage{booktabs}

\newtheorem{mydef}{Definition}[section]
\newtheorem{thm}{Theorem}[section]
\setcounter{section}{4}

\DeclareMathOperator*{\argmin}{arg\,min}
\DeclareMathOperator*{\argmax}{arg\,max}
\newcommand{\Lim}[1]{\raisebox{0.5ex}{\scalebox{0.8}{$\displaystyle \lim_{#1}\;$}}}
\newcommand{\E}{\mathbb{E}}
\newcommand{\Prob}{\mathbb{P}}
\newcommand{\V}{\text{Var}}
\newcommand{\iid}{\stackrel{iid}{\sim}}
\newcommand{\cblack}{\color{Black}}
\newcommand{\cblue}{\color{MidnightBlue}}

\providecommand{\tightlist}{%
  \setlength{\itemsep}{0pt}\setlength{\parskip}{0pt}}

\begin{document}

\justify

{\LARGE Handout 10: Likelihood-Ratio Test}

\vspace*{18pt}

\noindent
In addition to the nice point-estimator properties of MLE estimators,
they also come along with a built-in general procedure for building 
hypothesis tests. This process is called the \textbf{likelihood-ratio test};
while it only generates asymptotically correct tests in the limit of large
data, it often performs very well even for small datasets. It will be
helpful for us to a start with a slightly more theoretical description
of the test than we needed for the introductory methods we coved in the
first weeks.

Assume that we are working with a distribution that has $k$ unknown
parameters. We are going to define a null-hypothesis as a subset of
all possible configurations of the unknown parameters, which is in
turn a subset of $\mathbb{R}^k$. Symbolically, we have 
$\Theta_0 \subset \Theta \subseteq \mathbb{R}^k$. We can define the
G-score as the following:
\begin{align*}
G &= -2 \cdot \log \left[ \frac{
  \sup_{\theta \in \Theta_0} \mathcal{L}(\theta)
}{
  \sup_{\theta \in \Theta} \mathcal{L}(\theta)
} \right].
\end{align*}
The quantity inside of the brackets must be between zero and one, and
therefore $G$ will always be positive. If we define $\theta_0$ as the
argmax of the numerator and $\hat{\theta}$ as the argmax of the numerator,
this can be simplified in terms of the log-likelihood $l(\cdot)$:
\begin{align*}
G &= 2 \cdot \left[ l(\hat{\theta}) - l(\theta_0) \right].
\end{align*}
There is an important (but not easy to prove) result called
\textbf{Wilks' Theorem} that establishes that $G$ will be 
approximately distributed as a $\chi^2$. The degrees of freedom
will be the difference in dimensionality between $\Theta$ and
$\Theta_0$. 

A common case that the null hypothesis $\Theta_0$ is just a
single, fixed value. In this case, the distribution of $G$ will
be a chi-squared with degrees of freedom equal to the number of
unknown parameters. The case of a larger null-hypothesis is
often needed. We will see an example of this on today's worksheet.

\end{document}

