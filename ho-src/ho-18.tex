\documentclass{tufte-handout}

\usepackage{amssymb,amsmath}
% \usepackage{mathspec}
\usepackage{graphicx,grffile}
\usepackage{longtable}
\usepackage{booktabs}

\newtheorem{mydef}{Definition}[section]
\newtheorem{thm}{Theorem}[section]
\setcounter{section}{4}

\DeclareMathOperator*{\argmin}{arg\,min}
\DeclareMathOperator*{\argmax}{arg\,max}
\newcommand{\Lim}[1]{\raisebox{0.5ex}{\scalebox{0.8}{$\displaystyle \lim_{#1}\;$}}}
\newcommand{\E}{\mathbb{E}}
\newcommand{\Prob}{\mathbb{P}}
\newcommand{\V}{\text{Var}}
\newcommand{\iid}{\stackrel{iid}{\sim}}
\newcommand{\cblack}{\color{Black}}
\newcommand{\cblue}{\color{MidnightBlue}}

\providecommand{\tightlist}{%
  \setlength{\itemsep}{0pt}\setlength{\parskip}{0pt}}

\begin{document}

\justify

{\LARGE Handout 18: Bayesian Statistics II}

\vspace*{18pt}

\noindent
Today, we will once again consider the task of estimating the 
parameter $p$ from a random variable $X$ taken from a $Bin(n, p)$
distribution with a known value of $n$. The approach, though,
will be quite different than our previous attempts. 

The core novel idea behind Bayesian statistics can be summarized
as this: use probability distributions to model our uncertainty in
the unknown parameters of a distribution. So, say that before 
observing any data, we think that any value of $p$ is equally
likely. We could write this by defining our unknown parameter $P$
to be a random variable with a uniform distribution:
\begin{align*}
P \sim Unif(0, 1).
\end{align*}
This is called the \textbf{prior distribution}, because it reflects
our knowledge of $P$ prior to observing any data.\footnote{
  As I have with some other cases, I am going to introduce the
  terminology and notation of Bayesian estimation using a specific
  example. How to extend this to other cases should be clear after
  noting that the variables called $P$ and $X$ here may have different
  names in other cases.
}
Now, when describing
the random variable $X$, we have to give its distribution conditioned
on a specific value of the random variable $p$. That is, we need to
write this (which we call the \textbf{likelihood}, following the notation
from the MLE):
\begin{align*}
X|P \sim Bin(n, P).
\end{align*}
Now, the important thing is describing our knowledge about $P$ \textit{after}
observing the data. That is, we want to know the distribution of $P | X$.
Bayes rule tells us that we can calculate this as:
\begin{align*}
f_{P|X}(p|x) &= \frac{f_{X|P}(x|p) \times f_{P}(p)}{f_{X}(x)}.
\end{align*}
This quantity is called the \textbf{posterior distribution}. Determining
the form of the posterior distribution is the key task in generating
Bayesian estimators. One simplifying step is to notice that the denominator
does not depend on $p$, so we can replace it with a constant, adding it
back later (if needed) by whatever number makes the posterior a proper
distribution (in other words, it integrates to $1$). This gives the following
standard form:
\begin{align*}
f_{P|X}(p|x) &\propto \frac{f_{X|P}(x|p) \times f_{P}(p)}{f_{X}(x)}.
\end{align*}
I will try to keep the subscripts on the density functions $f$
for clarity in the notes. However, on the board I will almost always drop
them. Feel free to do the same in your work.

Now let's actually find the posterior distribution for this specific example.
We have the following form of the density function (keep in mind that this
is a function of p; we can remove any constants that depend only on $x$ and 
$n$):
\begin{align*}
f_{P|X}(p|X) &\propto \binom{n}{x} \cdot p^{x} \cdot (1 - p)^{n-x} \\
&\propto p^{x} \cdot (1 - p)^{1-x} \\
&=p^{(x+1) - 1} \cdot (1 - p)^{(n-x+1) - 1}
\end{align*}
The last step may seem unusual, but if you look at the distribution table 
it becomes more clear. This is a Beta distribution, with $\alpha = (x+1)$
and $\beta = (n-x+1)$. So, the posterior is given by:
\begin{align*}
P|X \sim Beta(x+1, n-x+1).
\end{align*}
This new distribution represents our knowledge and uncertainty about the
parameter $P$.\footnote{
  I have used a capital $P$ to stress that the parameter is now a
  random variable. While I think this is more clear, and I will use 
  it in the Binomial case in the worksheet, note that this is not a 
  standard notation. Usually we just use the same letter we have used
  throughout, with the change from constant to random variable being
  implicit (and sometimes confusing before you get the hang of it).
}

While the entire distribution is the clearest picture of our knowledge of
$P$, sometimes we need to convert our knowledge into a single best guess
point estimator. The \textbf{Bayesian point estimator} is the expected value
of the posterior distribution. So, using the table, here we have:
\begin{align*}
\hat{p}_{Bayes} &= \mathbb{E} [P|X] = \frac{x+1}{(x+1)+(n-x+1)} = \frac{x+1}{n + 2}.
\end{align*}
Notice that this limits the MLE in the limit of large $n$. Similarly, we
can represent a version of a confidence interval for Bayesian statistics.
A \textbf{credible interval} with credibility $1-\alpha$ for the parameter 
$P$ can be constructed by finding fixed values $l$ and $u$ such that:
\begin{align*}
\mathbb{P}[l \leq P \leq u] &= 1 - \alpha.
\end{align*}
We can compute these values using R, as demonstrated in the notebook for 
today. 

\vspace*{24pt}

\noindent
There are many methodological and philosophical implications of using
Bayesian methods in place of the frequentist techniques that we have
so far used. I hope that we can discuss these more in the classes to
come after we have some more experience with the mechanics of how to
work with them computationally.

\end{document}



