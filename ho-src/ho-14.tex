\documentclass{tufte-handout}

\usepackage{amssymb,amsmath}
% \usepackage{mathspec}
\usepackage{graphicx,grffile}
\usepackage{longtable}
\usepackage{booktabs}

\newtheorem{mydef}{Definition}[section]
\newtheorem{thm}{Theorem}[section]
\setcounter{section}{4}

\DeclareMathOperator*{\argmin}{arg\,min}
\DeclareMathOperator*{\argmax}{arg\,max}
\newcommand{\Lim}[1]{\raisebox{0.5ex}{\scalebox{0.8}{$\displaystyle \lim_{#1}\;$}}}
\newcommand{\E}{\mathbb{E}}
\newcommand{\Prob}{\mathbb{P}}
\newcommand{\V}{\text{Var}}
\newcommand{\iid}{\stackrel{iid}{\sim}}
\newcommand{\cblack}{\color{Black}}
\newcommand{\cblue}{\color{MidnightBlue}}

\providecommand{\tightlist}{%
  \setlength{\itemsep}{0pt}\setlength{\parskip}{0pt}}

\begin{document}

\justify

{\LARGE Handout 14: Simple Linear Regression}

\vspace*{18pt}

\noindent
\underline{Setup}
Let $x_1, \ldots, x_n$ be a set of fixed real numbers.
Consider observing an independent random sample of size $n$ denoted
by $Y_1, \ldots, Y_n$ where
\begin{align*}
Y_i \sim N(b_0 + b_1 \cdot x_i, \, \sigma^2)
\end{align*}
For some unknown constants $b_0$, $b_1$, and $\sigma^2$. So, while
the observations are independent, they are not identically distributed
because they have different means.
This model is called a \textbf{(simple) linear regression}.\footnote{
  We will see the more general case of a linear regression in the next
  set of notes.
}
One can visualize the model as trying to fit a line with intercept $b_0$
and slope $b_1$ to a plot with the $x_i$'s on the x-axis and the $Y_i$'s on
the y-axis. It is often called a \textit{best-fit line} in a non-technical
context.

\vspace*{12pt}

\noindent
\underline{Interpretation}
The slope parameter $b_1$ has an important interpretation: it gives
the amount that we expect $Y$ to change on average with a unit change
in $x$. For example, if $x$ are the number of hours you prepare for the
SAT and $Y$ is your score on the exam, $b_1$ would be the average
increase in SAT score for every extra hour studied.

\vspace*{12pt}

\noindent
\underline{Point Estimators}
As you might imagine, the most popular method for estimating the 
unknown parameters is maximum likelihood. On today's worksheet, we
will derive the MLE in the special case that $b_0$ is equal to zero.
In the full version, the MLE values for the intercept and slope can
be computed as
\begin{align*}
\widehat{b}_1 &= \frac{\sum_i (x_i - \bar{x}) \cdot (Y_i - \bar{Y})}{\sum_i (x_i - \bar{x})^2} \\
\widehat{b}_0 &= \bar{Y} -  \widehat{b}_1 \cdot \bar{x}
\end{align*}
If we define $\widehat{Y}_i$ to be $\widehat{b}_0 + \widehat{b}_1 \cdot x_i$,
the \textbf{fitted value} of $Y$, then we can write an unbiased 
variation of the MLE of the variance as:\footnote{
  The actual MLE will have a denominator of $n$, but this will be
  biased, which we can fix by multiplying by $n/(n-2)$, which is
  usually done.
}
\begin{align*}
\widehat{\sigma^2} &= \frac{1}{n-2} \sum_i (Y_i - \widehat{Y}_i)^2.
\end{align*}
There should be some intuition from our previous results about the
denominator in the previous equation from the fact that we have
$n$ data points and a model for the mean with $2$ parameters. The
difference in their dimensionality gives the expected amount that the
predicted values of $Y$ should vary about their actual values.

\vspace*{12pt}

\noindent
\underline{Inference}
The form of the estimators for the slope and intercept allow us to
compute explicit T-statistics much like we did for the first unit of
the course. You will see how to do this is the simple case on the 
worksheet and in the more general case on the next handout.

\end{document}



