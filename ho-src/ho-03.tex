\documentclass{tufte-handout}

\usepackage{amssymb,amsmath}
% \usepackage{mathspec}
\usepackage{graphicx,grffile}
\usepackage{longtable}
\usepackage{booktabs}

\newtheorem{mydef}{Definition}[section]
\newtheorem{thm}{Theorem}[section]
\setcounter{section}{3}

\DeclareMathOperator*{\argmin}{arg\,min}
\DeclareMathOperator*{\argmax}{arg\,max}
\newcommand{\Lim}[1]{\raisebox{0.5ex}{\scalebox{0.8}{$\displaystyle \lim_{#1}\;$}}}
\newcommand{\E}{\mathbb{E}}
\newcommand{\Prob}{\mathbb{P}}
\newcommand{\V}{\text{Var}}
\newcommand{\iid}{\stackrel{iid}{\sim}}
\newcommand{\cblack}{\color{Black}}
\newcommand{\cblue}{\color{MidnightBlue}}

\providecommand{\tightlist}{%
  \setlength{\itemsep}{0pt}\setlength{\parskip}{0pt}}

\begin{document}

\justify

{\LARGE Handout 03: Confidence Intervals}

\vspace*{18pt}

\noindent
Let $\theta$ be a quantity of interest that we are trying to estimate from a
random sample drawn from a distribution $\mathcal{G}$. A \textbf{confidence interval}
with \textbf{confidence level} $(1-\alpha)$ is a pair of sample statistics
$L$ and $U$ such that:
\begin{align*}
\Prob\left[ L \leq \theta \leq U \right] \geq 1 - \alpha.
\end{align*}
The idea is that we want to have a high probability that the quantity of interest
falls between the lower bound $L$ and upper bound $U$.

A standard approach to deriving a confidence interval is to start with a random
random variable called a \textbf{pivot}. A pivot is defined as a function of the
random sample and parameters defining the population $\mathcal{G}$ whose distribution
does not depend on the unknown parameters. Let's walk through an example where $\mathcal{G}$
is equal to $N(\theta, 1)$ with an unknown mean $\theta$. The following value is a
pivot because, as we have written, it will have a standard normal distribution regardless
of the value of $\theta$:
\begin{align*}
Z &= \frac{\theta - \bar{X}}{\sqrt{1/n}} \sim N(0, 1).
\end{align*}
Since we know the distribution of $Z$, we can write something that looks like 
a confidence interval for a given confidence level. For example, with $\alpha = 0.01$,
we have:
\begin{align*}
\Prob\left[ -2.58 \leq Z \leq 2.58 \right] &\approx 0.99 = 1 - 0.01 \\
\Prob\left[ -2.58 \leq \frac{\theta - \bar{X}}{\sqrt{1/n}} \leq 2.58 \right] &\approx 0.99
\end{align*}
To get the actual confidence interval, we manipulate the part inside the probability
so that the parameter $\theta$ is alone in the middle and the lower and upper bounds
depend only on the random sample:
\begin{align*}
\Prob\left[ \bar{X} - 2.58 \cdot \sqrt{1/n} \leq \theta \leq \bar{X} + 2.58 \cdot \sqrt{1/n} \right] &\approx 0.99
\end{align*}
We see that we can get a confidence interval by picking something centered on the sample
mean with length $2 \cdot 2.58 \cdot \sqrt{1/n}$.

A handy notation for defining formulae for confidence intervals is to define $z_{\alpha}$
to be the following quantity:
\begin{align*}
\Prob[Z \leq z_{\alpha}] &= \alpha, \quad Z \sim N(0, 1).
\end{align*}
We will also define analogous quantities $t_\alpha(k)$ and $\chi_\alpha^2(k)$ for the
t-distribution and chi-squared distributions. Replacing this with the $\pm2.58$ above, we
have the more general formula:
\begin{align*}
\Prob\left[ \bar{X} + z_{\alpha/2} \cdot \sqrt{1/n} \leq \frac{\theta - \bar{X}}{\sqrt{1/n}} \leq \bar{X} + z_{1 - \alpha/2} \cdot \sqrt{1/n} \right] &\approx 1 - \alpha
\end{align*}
The confidence interval above is valid for any confidence level $\alpha$. Because the
normal distribution is symmetric around the origin, we have that
$z_{\alpha/2} = - z_{1 - \alpha/2}$. This means that you could rewrite the confidence
interval as:
\begin{align*}
\Prob\left[ \bar{X} - z_{1 - \alpha/2} \cdot \sqrt{1/n} \leq \frac{\theta - \bar{X}}{\sqrt{1/n}} \leq \bar{X} + z_{1 - \alpha/2} \cdot \sqrt{1/n} \right] &\approx 1 - \alpha
\end{align*}
Or even:
\begin{align*}
\bar{X} \pm z_{1 - \alpha/2} \times \sqrt{\frac{1}{n}}.
\end{align*}
The latter is a common way of writting a confidence interval for a mean.

The example above is quite artificial because it assumes that we already known the variance
of $\mathcal{G}$. On today's worksheet, we will see that the following is a pivot statistic
in the general case where the distribution is $N(\mu, \sigma^2)$:
\begin{align*}
T &= \frac{\mu - \bar{X}}{\sqrt{S_X^2 / n}} \sim t(n-1).
\end{align*}
Importantly, it can also be used as an approximation for any $\mathcal{G}$ with finite mean
and variance for large $n$ due to the central limit theorem. For reference, here is the
confidence interval that we will be deriving:
\begin{align*}
\bar{X} \pm t_{1 - \alpha/2} \times \sqrt{\frac{S_X^2}{n}}.
\end{align*}
In addition to being a helpful formula to have as a computational tool, this quantity also
helps conceptualize how our ability to estimate a mean scales with the desired confidence
($\alpha$), the variation in the data ($S_X^2$), and the sample size ($n$).

\end{document}

