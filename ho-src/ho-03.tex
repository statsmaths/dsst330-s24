\documentclass{tufte-handout}

\usepackage{amssymb,amsmath}
% \usepackage{mathspec}
\usepackage{graphicx,grffile}
\usepackage{longtable}
\usepackage{booktabs}

\newtheorem{mydef}{Definition}[section]
\newtheorem{thm}{Theorem}[section]
\setcounter{section}{3}

\DeclareMathOperator*{\argmin}{arg\,min}
\DeclareMathOperator*{\argmax}{arg\,max}
\newcommand{\Lim}[1]{\raisebox{0.5ex}{\scalebox{0.8}{$\displaystyle \lim_{#1}\;$}}}
\newcommand{\E}{\mathbb{E}}
\newcommand{\Prob}{\mathbb{P}}
\newcommand{\V}{\text{Var}}
\newcommand{\iid}{\stackrel{iid}{\sim}}
\newcommand{\cblack}{\color{Black}}
\newcommand{\cblue}{\color{MidnightBlue}}

\providecommand{\tightlist}{%
  \setlength{\itemsep}{0pt}\setlength{\parskip}{0pt}}

\begin{document}

\justify

{\LARGE Handout 03: Confidence Intervals}

\vspace*{18pt}

\noindent
\underline{Definition and Derivation}\\
Let $\theta$ be a quantity of interest that we are trying to estimate from a
random sample drawn from a distribution $\mathcal{G}$. A \textbf{confidence interval}
with \textbf{confidence level} $(1-\alpha)$ is a pair of sample statistics
$L$ and $U$ such that:
\begin{align*}
\Prob\left[ L \leq \theta \leq U \right] \geq 1 - \alpha.
\end{align*}
The idea is that we want to have a high probability that the quantity of interest
falls between the lower bound $L$ and upper bound $U$.

A standard approach to deriving a confidence interval is to start with a random
random variable called a \textbf{pivot}. A pivot is defined as a function of the
random sample and parameters defining the population $\mathcal{G}$ whose distribution
does not depend on the unknown parameters. Let's walk through an example where
$\mathcal{G}$ is equal to $N(\mu_X, 1)$ with an unknown mean $\mu_X$. The following
value is a pivot because, as we have written, it will have a standard normal
distribution regardless of the value of $\mu_X$:
\begin{align*}
Z &= \frac{\mu_X - \bar{X}}{\sqrt{1/n}} \sim N(0, 1).
\end{align*}
Since we know the distribution of $Z$, we can write something that looks like 
a confidence interval for a given confidence level. For example, with $\alpha = 0.01$,
we have:\footnote{
  The first line is something that we would need to lookup on a table
  or use software such as R to compute. We will see that next class.
}
\begin{align*}
\Prob\left[ -2.58 \leq Z \leq 2.58 \right] &\approx 0.99 = 1 - 0.01 \\
\Prob\left[ -2.58 \leq \frac{\mu_X - \bar{X}}{\sqrt{1/n}} \leq 2.58 \right] &\approx 0.99
\end{align*}
To get the actual confidence interval, we manipulate the part inside the probability
so that the parameter $\mu_X$ is alone in the middle and the lower and upper bounds
depend only on the random sample:
\begin{align*}
\Prob\left[ \bar{X} - 2.58 \cdot \sqrt{1/n} \leq \mu_X \leq \bar{X} + 2.58 \cdot \sqrt{1/n} \right] &\approx 0.99
\end{align*}
And that's really it! We now have a confidence intervale for the unknown
mean parameter $\mu_X$. One slightly more compact way to write this is 
as the following:
\begin{align*}
\bar{X} \pm 2.58 \cdot \sqrt{1/n}.
\end{align*}
We could also write it down as an interval with the lower and upper bound,
which looks nice when we are working with specific numbers.

\newpage

\noindent
\underline{Notation for Tail Bounds} \\
A handy notation for defining formulae for confidence intervals would be to
replace the constant $2.58$ with a more general term that we can fill in based
on the confidence level. To do this, let $z_{\alpha}$ be the following
quantity:\footnote{
  Note that some sources will give the probability in the other direction,
  or possibly only one sided probabilities. Ideas are all the same, but the
  specifics you find elsewhere may be a $\pm$ sign off from our results.
}
\begin{align*}
\Prob[z_{\alpha} \geq Z ] &= \alpha, \quad Z \sim N(0, 1).
\end{align*}
This gives the tail probability that a standard normal is greater than the value
$z_{\alpha}$. Since the normal is symmetric around the origin, we can use 
$\pm z_{\alpha/2}$ as the parameter for a confidence interval. That is, we 
have the problem above, we would have the a confidence interval with 
confidence level $(1-\alpha)$ given by the following:
\begin{align*}
\bar{X} \pm z_{\alpha/2} \cdot \sqrt{1/n}.
\end{align*}
We will also define analogous quantities $\chi_\alpha^2(k)$, $f_\alpha(d_1, d_2)$,
and $t_\alpha(k)$ for the chi-squared, F-, and t-distributions.

\vspace*{20pt}

\noindent
\underline{T-Statistic} \\
Assume that we have two independent random variables: $Z \sim N(0, 1)$ and
$C \sim \chi^2(k)$. Then, define the following ratio between the two random
variables:
\begin{align*}
T &= \frac{Z}{\sqrt{C / k}}.
\end{align*} 
It should be clear that this random variable has a well defined distribution that
depends only on the degrees of freedom $k$ of the chi-squared distribution. The
distribution is called \textbf{Student's t-distribution} with $k$ degrees of freedom,
which is denoted by $t(k)$. This will be important for our work today to derive the
standard confidence interval for the mean of an unknown distribution.

This will be useful combined with the facts we showed last week. Namely,
that  if $\mathcal{G}$ is a normal distribution, then we have the following:
\begin{align*}
\frac{\mu_X - \bar{X}}{\sqrt{\sigma_X^2 / n}}  &\sim N(0, 1) \\
\frac{(n-1)S^2_X}{\sigma_X^2} &\sim \chi^2(n-1).
\end{align*}
We will create a pivot statistic based on these two quantities and then derive
a confidence interval for the mean.

% The example above is quite artificial because it assumes that we already known the variance
% of $\mathcal{G}$. On today's worksheet, we will see that the following is a pivot statistic
% in the general case where the distribution is $N(\mu, \sigma^2)$:
% \begin{align*}
% T &= \frac{\mu - \bar{X}}{\sqrt{S_X^2 / n}} \sim t(n-1).
% \end{align*}
% Importantly, it can also be used as an approximation for any $\mathcal{G}$ with finite mean
% and variance for large $n$ due to the central limit theorem. For reference, here is the
% confidence interval that we will be deriving:
% \begin{align*}
% \bar{X} \pm t_{1 - \alpha/2} \times \sqrt{\frac{S_X^2}{n}}.
% \end{align*}
% In addition to being a helpful formula to have as a computational tool, this quantity also
% helps conceptualize how our ability to estimate a mean scales with the desired confidence
% ($\alpha$), the variation in the data ($S_X^2$), and the sample size ($n$).

\end{document}

