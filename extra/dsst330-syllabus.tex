\documentclass[11pt, a4paper]{article}

\usepackage{fontspec}
\usepackage{geometry}
\usepackage{fancyhdr}
\usepackage[hidelinks]{hyperref}
\usepackage[normalem]{ulem}
\usepackage{multicol}

\geometry{
  top=2cm,
  bottom=3cm,
  left=3cm,
  right=3cm,
  marginparsep=4pt,
  marginparwidth=1cm
}

\renewcommand{\headrulewidth}{0pt}
\pagestyle{fancyplain}
\fancyhf{}
\lfoot{}
\rfoot{}

\setlength{\parindent}{0pt}
\setlength{\parskip}{0pt}

\usepackage{xunicode}
\defaultfontfeatures{Mapping=tex-text}

\begin{document}

\begin{center}
\textbf{DSST 330: Mathematical Statistics --- Taylor Arnold --- Spring 2024}
\end{center}

\vspace{0.5cm}

\textbf{Website}: \texttt{https://statsmaths.github.io/dsst330-s24}

\bigskip

\textbf{Topics:}
This course presents statistical modeling using the language of
probability theory. Rather than focusing on theoretical results for their
own sake, we concentrate on examples in which probability theory
provides better approaches and a deeper understanding of statistical
models than would be available in a non-probability based course.
For each model, we will see how to use the R programming language to
apply the technique and will show a real application in which the model
can be applied. Specific topics will include Bayesian estimation,
the method of moments, credible/confidence intervals, and multivariate
linear regression.

\bigskip

\textbf{Prerequisites:}
We will assume a general knowledge of random variables, moment generating
functions, and other standard topics from a calculus-based probability
course (such as MATH/DSST329). No prior experience with statistics or the
R programming language is required. However, some prior knowledge will help
to best motivate the examples.

\bigskip

\textbf{General Format:}
Most class meetings will start by going through a handout that
introduces new material followed by working together through a set of
exercises that take the form short programming tasks, data collection,
and/or mathematical derivations.

\bigskip

\textbf{Grades:}
Three cumulative numeric scores will be given throughout the semester:
an engagement/participation grade (out of 115), an exam grade (out of 120),
and a score for the final project (out of 100). Details are given in the
corresponding sections below. The final numeric grade will come from your
\textit{minimum} score across these three marks. A letter grade will be
assigned as follows:
             A (93+), A- (90--92),
B+ (87--89), B (83--86), B- (80--82),
C+ (77--79), C (73--76), C- (70--72), and F (0--69).

\bigskip

\textbf{Class Engagement:}
Everyone is expected to regularly attend coure meetings and actively
participate. Unless otherwise noted, please bring your laptop, a pen
or pencil, and all previous course handouts to each class meeting.
Attendence will be taken with a sign-in sheet. Each of the twenty three
non-exam and non-first week classes are worth a total of five points. In
the event that a class meeting is canceled, everyone will be given full
marks. In order to ensure a full five points, please arrive by the start
of class, have all of your material ready, and participate in all of the
class activities.

\bigskip

\textbf{Exams:}
There are three exams, each of which may have both take-home (open book
and resources) and in-class (closed-book) components. Each exam will be
worth 40 points, split evenly between the two parts. Tentative exam dates
are given on the course website. There is no final exam.

\bigskip

\textbf{Final Project:}
The course has a final project component, which will be presented during
the last week of the course. Each student will create materials explaining
either a statistical estimator or application that has not been covered
during the semester. These will be posted on the course website for
reference.

\bigskip

\textbf{Office Hours:}
There will be significant time during class to ask questions about
the material. Please email to ask additional questions or schedule a
one-on-one meeting at any point during the semester. Keep in
mind that office hours should supplement rather than replace
attending and asking questions during class.

\end{document}