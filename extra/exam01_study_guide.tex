\documentclass{tufte-handout}

\usepackage{amssymb,amsmath}
% \usepackage{mathspec}
\usepackage{graphicx,grffile}
\usepackage{longtable}
\usepackage{booktabs}

\newtheorem{mydef}{Definition}[section]
\newtheorem{thm}{Theorem}[section]
\setcounter{section}{1}

\providecommand{\tightlist}{%
  \setlength{\itemsep}{0pt}\setlength{\parskip}{0pt}}

\begin{document}

\justify

{\LARGE Exam 01: Study Guide}

\vspace*{18pt}

\noindent
The exam has two parts with equal weights, an open-notes
take-home part and a closed-notes in-class part.

\vspace*{18pt}

\noindent
The take-home
portion will consist of a small data collection exercise and
data analysis where you apply one or more of the techniques that
we have learned so far. You are welcome to use R (and I will 
even provide a short template to get you started), but you
can also do the calculations by hand. The only part you hand
in will be a paper with a short descriptions and your results.
I will distribute the instructions at the end of class on Wednesday.

\vspace*{18pt}

\noindent
After some reflection, I have decided to change the format for
the first exam. Rather than having you replicate the theoretical
calculations we did in class, the questions here will focus on
the application of those results to data. I will give you several
different small datasets and/or summary statistics (such as $\bar{X}$
or $S_X^2$) and will ask you to produce the correct point estimators,
confidence intervals, or hypothesis tests for a given situation. Any
critical values you need (i.e., $t_{\alpha}$) will be provided.
\underline{Bring a calculator}. The exam should be easy as long as
you are able to derive and/or memorize the table of common statistical
tests.

\end{document}

