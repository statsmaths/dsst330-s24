\documentclass{tufte-handout}

\usepackage{amssymb,amsmath}
% \usepackage{mathspec}
\usepackage{graphicx,grffile}
\usepackage{longtable}
\usepackage{booktabs}

\newtheorem{mydef}{Definition}[section]
\newtheorem{thm}{Theorem}[section]
\setcounter{section}{1}

\providecommand{\tightlist}{%
  \setlength{\itemsep}{0pt}\setlength{\parskip}{0pt}}

\begin{document}

\justify

{\LARGE Final Project Instructions}

\vspace*{18pt}

\noindent
There are two goals for the final project. First, to teach you how
to learn about and apply new statistical models beyond those covered
this semester. We have learned some of the most important techniques
and general approaches, but there are hundreds of other tests and
models beyond those that could fit into any course. However, you should
have the skills now to be able to teach yourself new techniques
with a little bit of work. Secondly, while we will still will not
come even close to covering all the models you might see, watching
all of the other presentations will give you better sense of the 
diversity of different models that extend from the core ones we
learned in class.

\vspace*{18pt}

\noindent
We will be selecting presentation times and topics in class on
Wednesday, March 27th. The main outcome of the project will be a
10 minute presentation to the class on one of the final days of the
semester. While the specific format is up to you, there must be some
sort of digital presentation that is detailed enough to follow along
without your presentation. I recommend creating slides, an R Markdown
document similar to the notes from class, or a set of LaTeX notes 
similar to the handouts. The best format will depend on your topic
and level of comfort with each of these technologies. You will need
to hand in a PDF version of whatever format you choose; this will not
be published on the website, but will be accessible to others in the
class.

\vspace*{18pt}

\noindent
Your presentation should cover at a minimum the following aspects of
your topic: (1) general motivation for the idea/model/approach, (2)
a mathematical description of the model, (3) one or more authoratitive
references justifying the model, (4) an R package or function that 
impliments the models, and (5) an example applying the model to a real,
appropriately selected dataset. You do not need to show proofs or
derivations of why the method works, but can include these details (or
a sketch of them) if the result is relatively short.

\vspace*{18pt}

\noindent
Grades will be largely determined based on effort, polish, and an 
earnest attempt at the completion of all aspects of the assignment
as listed above. Start early, make use of the class time to work on
the project, and practice your presentation and you will do great.
Note that, out of respect for your fellow classmates, everyone
\textbf{must attend and be on-time} for all presentation days.

\end{document}

