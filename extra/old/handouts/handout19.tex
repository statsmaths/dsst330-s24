\documentclass{tufte-handout}

\usepackage{amssymb,amsmath}
% \usepackage{mathspec}
\usepackage{graphicx,grffile}
\usepackage{longtable}
\usepackage{booktabs}

\newtheorem{mydef}{Definition}
\newtheorem{thm}{Theorem}
\DeclareMathOperator*{\argmin}{arg\,min}
\DeclareMathOperator*{\argmax}{arg\,max}
\newcommand{\iid}{\stackrel{iid}{\sim}}

\providecommand{\tightlist}{%
  \setlength{\itemsep}{0pt}\setlength{\parskip}{0pt}}

\setlength{\parskip}{12pt}
\setlength\parindent{0pt}

\begin{document}

\justify

{\LARGE Handout 19: Confidence Intervals}

\vspace*{18pt}

\noindent

Over the past few classes we have discussed properties that make
for good estimators $\widehat{\theta}$ of some unknown parameter
$\theta$. These are often called \textit{point estimators} as they
specify the single best guess (i.e., a point) for $\theta$. Today
we will discuss providing a range of guesses for a parameter value;
this is called a \textit{confidence interval}. Confidence intervals
make probabilistic guarantees about how likely it is that $\theta$
is contained in the predicted interval. That is:
\begin{align*}
\mathbb{P} \left[ \widehat{\theta}_{lower} \leq \theta \leq \widehat{\theta}_{upper} \right] \geq 1 - \alpha
\end{align*}
For some reasonable $\alpha$ near $0$.

Suppose that $X_1, X_2, \ldots, X_n \iid F$ with a finite
mean $\mu$ and finite variance $\sigma^2$. Consider
the following statistic:
\begin{align*}
t &= \frac{\bar{X} - \mu}{s / \sqrt{n}}
\end{align*}
Where, as before,
\begin{align*}
s &= \frac{1}{n-1} \sum_{i=1}^n (X_i - (\bar{X}))^2.
\end{align*}
In this case we know by the central limit theorem that $t$
is asymptotically distributed as $N(0,1)$.\footnote{Because the
distribution does not depend on the parameter of interest, $\mu$,
we say that it $t$ a pivot statistic.}
Define $z_{\alpha}$ as the $\alpha$ cutoff for the
standard normal distribution,
\begin{align*}
\alpha &= \int_{z_\alpha}^\infty \frac{1}{\sqrt{2\pi}} e^{-\frac{x^2}{2}} dx,
\end{align*}
We then have:
\begin{align*}
\mathbb{P} \left( -z_{\alpha/2} \leq \frac{\bar{X} - \mu}{s / \sqrt{n}} \leq z_{\alpha/2} \right) &\approx 1 - \alpha.
\end{align*}
Which can be re-written as:
\begin{align*}
\mathbb{P} \left( \bar{X} - (s / \sqrt{n}) z_{\alpha/2} \leq \mu \leq \bar{X} +(s / \sqrt{n}) z_{\alpha/2} \right) &\approx 1 - \alpha
\end{align*}
Or, even as:
\begin{align*}
\bar{X} \pm \frac{s}{\sqrt{n}} \cdot z_{\alpha/2}
\end{align*}
This particular case is important because we can use it without knowing
the exact distribution family that $X$ is drawn from. Therefore, we can
use this formula to approximately capture the mean from any sample taken
independently from a distribution with finite variance.

\end{document}









