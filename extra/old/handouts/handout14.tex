\documentclass{tufte-handout}

\usepackage{amssymb,amsmath}
% \usepackage{mathspec}
\usepackage{graphicx,grffile}
\usepackage{longtable}
\usepackage{booktabs}

\newtheorem{mydef}{Definition}
\newtheorem{thm}{Theorem}
\DeclareMathOperator*{\argmin}{arg\,min}
\DeclareMathOperator*{\argmax}{arg\,max}
\newcommand{\iid}{\stackrel{iid}{\sim}}

\providecommand{\tightlist}{%
  \setlength{\itemsep}{0pt}\setlength{\parskip}{0pt}}

\setlength{\parskip}{12pt}
\setlength\parindent{0pt}

\begin{document}

\justify

{\LARGE Handout 14: Unbiasedness and MSE}

\vspace*{18pt}

\noindent

We begin making the transition from the ideas and tools
of probability theory to the theory and methods of statistical
inference. For simplicity, suppose that we have the data
$X_1, X_2, \ldots, X_n$ in hand,
where the X's are assumed to be independent random variables with a
common distribution $F_\theta$ indexed by the parameter $\theta$
whose value is unspecified. In practice, we would often be willing to
assume that the distribution $F_\theta$ has a density or
probability mass function of some known form. For example, physical measurements
might be assumed to be random draws from an $N(\mu, \sigma^2)$
distribution, counts might be assumed to be
$Binomial(n, p)$ or $Poisson(\lambda)$ variables, and the failure times of
engineered systems in a life-testing experiment might be assumed to be
$\Gamma(\alpha, \beta)$ variables. Note that such assumptions do not
specify the exact distribution that applies to the experiment of
interest, but rather, specifies only its ``type.''

Suppose that a particular parametric model \emph{F}\textsubscript{$\theta$ } has
been assumed to apply to the available data \emph{X}\textsubscript{1},
\emph{X}\textsubscript{2},\ldots{}, \emph{X\textsubscript{n}}. It is
virtually always the case in practice that the exact value of the
parameter $\theta$  is not known. A naïve but nonetheless useful way to think of
\emph{statistical estimation} is to equate it with the process of
guessing. The goal of statistical estimation is to make an
\emph{educated guess} about the value of the unknown parameter $\theta$ . What
makes one's guess ``educated'' is the fact that the estimate of $\theta$  is
informed by the data. A \emph{point estimator} of $\theta$  is a fully specified
function of the data which, when the data is revealed, yields a
numerical guess at the value of $\theta$ . The estimator will typically be
represented by the symbol $\widehat{\theta}$,
pronounced ``theta hat,'' whose dependence on the data is reflected in
the equation:
\begin{align*}
\widehat{\theta} = \widehat{\theta}(X_1, X_2, \ldots, X_n)
\end{align*}

Let us suppose, for example, that the experiment of interest involves
\emph{n} independent tosses of a bent coin, where \emph{n} is a known
integer. The parameter \emph{p}, the probability of heads in a given
coin toss, is treated as an unknown constant. It seems quite reasonable
to assume that the experimental data we will observe is well described
as a sequence of iid Bernoulli trials. Most people would base their
estimate of \emph{p} on the variable \emph{X}, the number of successes
in \emph{n} iid Bernoulli trials; of course, \emph{X} \textasciitilde{}
\emph{B}(\emph{n, p}). The sample proportion of successes, $\widehat{p} = X /n$
is a natural estimator of \emph{p} and is, in fact, the estimator of
\emph{p} that most people would use.

What makes a ``good'' estimator? Ideally, we would hope that it tends
to be close to the true parameter it is estimating. The following defintion
formalizes this notion.

\begin{mydef}[Mean Squared Error (MSE)]
The mean squared error of an estimator $\widehat{\theta}$ of the parameter
$\theta$ is defined as:
\begin{align*}
MSE(\widehat{\theta}) &= \mathbb{E} \left( \widehat{\theta} - \theta  \right)^2
\end{align*}
\end{mydef}

The MSE has a useful decomposition that we will often use in the
analyses of various estimators.

\begin{thm}[Decomposition of MSE]
The mean squared error of $\widehat{\theta}$ can be decomposed as
follows:
\begin{align*}
MSE(\widehat{\theta}) &= Var(\widehat{\theta}) + Bias(\widehat{\theta})^2
\end{align*}
Where:
\begin{align*}
Bias(\widehat{\theta}) &= \mathbb{E} (\widehat{\theta} - \theta)
\end{align*}
\end{thm}
\textbf{Proof.} This result can be established with the following:
\begin{align*}
\widehat{\theta} - \theta &= (\widehat{\theta} - \mathbb{E}\widehat{\theta}) +
                              (\mathbb{E}\widehat{\theta} - \theta) \\
(\widehat{\theta} - \theta)^2 &=  (\widehat{\theta} - \mathbb{E}\widehat{\theta})^2 +
                                  (\mathbb{E}\widehat{\theta} - \theta)^2 +
                                  2 \cdot (\widehat{\theta} - \mathbb{E}\widehat{\theta})
                                  \cdot (\mathbb{E}\widehat{\theta} - \theta)
\end{align*}
Taking expectations on both sides, notice that the cross-term drops out:
\begin{align*}
\mathbb{E} (\widehat{\theta} - \theta)^2 &= \mathbb{E} (\widehat{\theta} - \mathbb{E}\widehat{\theta})^2 + \mathbb{E}(\mathbb{E}\widehat{\theta} - \theta)^2 \\
MSE(\widehat{\theta}) &= Var(\widehat{\theta}) + Bias(\widehat{\theta})^2
\end{align*}
Giving the desired result~$\blacksquare$.

Notice that the MSE in general is a function of the value of actual value
of $\theta$. The decomposition leads to a definition of bias:

\begin{mydef}[Unbiasedness]
An estimator $\widehat{\theta}$ is unbiased for $\theta$ if
$Bias(\widehat{\theta}) = 0$.
\end{mydef}

We close with a result that establishes the unbiasness of estimators
for the mean and standard deviation of an arbitrary estimator.

\begin{thm}
If $X_1, \ldots X_n \iid F$ for some distribution $F$ with
finite mean $\mu$ and variance $\sigma^2$, then
\begin{align*}
\bar{X} &= \frac{\sum_i X_i}{n} \\
s^2 &= \frac{1}{(n-1)} \sum_i (X_i - \bar{X})^2
\end{align*}
Are unbiased estimators of $\mu$ and $\sigma^2$, respectively.
\end{thm}
\textbf{Proof.} The first result is a simple application of the linearity
of the expectation operator. The second comes from a similar derivation
to the decomposition of the MSE:
\begin{align*}
\mathbb{E} \left[ \sum_{i=1}^n (X_i - \bar{X})^2 \right]
  &= \mathbb{E} \left[ \sum_{i=1}^n (X_i - \mu + \mu  - \bar{X})^2 \right] \\
  &= \mathbb{E} \left[ \sum_{i=1}^n (X_i - \mu)^2 + 2 (\mu - \bar{X}) \sum_{i=1}^n (X_i - \mu) + n (\bar{X} - \mu)^2 \right] \\
  &= \mathbb{E} \left[ \sum_{i=1}^n (X_i - \mu)^2 - 2 n (\bar{X} - \mu)^2 + n (\bar{X} - \mu)^2 \right] \\
  &= \mathbb{E} \left[ \sum_{i=1}^n (X_i - \mu)^2 - n (\bar{X} - \mu)^2 \right] \\
  &= \sum_{i=1}^n \mathbb{E}(X_i - \mu)^2 - n \mathbb{E} (\bar{X} - \mu)^2 \\
  &= \sum_{i=1}^n \sigma^2 - n \left( \frac{\sigma^2}{n} \right) \\
  &= (n-1)\sigma^2
\end{align*}
Dividing both sides by $(n-1)$ finishes the proof~$\blacksquare$.





\end{document}









