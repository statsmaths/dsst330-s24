\documentclass{tufte-handout}

\usepackage{amssymb,amsmath}
% \usepackage{mathspec}
\usepackage{graphicx,grffile}
\usepackage{longtable}
\usepackage{booktabs}

\newtheorem{mydef}{Definition}
\newtheorem{thm}{Theorem}
\DeclareMathOperator*{\argmin}{arg\,min}
\DeclareMathOperator*{\argmax}{arg\,max}
\newcommand{\iid}{\stackrel{iid}{\sim}}

\providecommand{\tightlist}{%
  \setlength{\itemsep}{0pt}\setlength{\parskip}{0pt}}

\setlength{\parskip}{12pt}
\setlength\parindent{0pt}

\begin{document}

\justify

{\LARGE Handout 18: Z-Scores}

\vspace*{6pt}

\noindent

We have mentioned that there is no closed form solution for
computing the following integral for arbitrary values $a$ and $b$:
\begin{align*}
\frac{1}{\sqrt{2 \pi \sigma^2}} \cdot \int_{a}^{b} e^{-\frac{1}{2}(x - \mu)^2}
\end{align*}
This integral come from taking $X \sim N(\mu, \sigma^2)$, and
finding the probability:
\begin{align*}
\mathbb{P}\left[ a \leq X \leq b \right].
\end{align*}
Given the important of the normal distribution, it seems that
we should have a way of approximating such probabilities. The
solution is to calculate a $Z$ score, which is a centered and
standardized version of random variable:
\begin{align*}
Z &= \frac{X - \mu}{\sigma}
\end{align*}
Notice that then we have $Z \sim N(0, 1)$. If we want to find
the probability that $X$ is between $a$ and $b$, we can write
this is terms of Z:
\begin{align*}
\mathbb{P} \left[ a \leq X \leq b  \right]
&= \mathbb{P} \left[ a - \mu \leq X - \mu \leq b - \mu  \right] \\
&= \mathbb{P} \left[ \frac{a - \mu}{\sigma} \leq \frac{X - \mu}{\sigma} \leq \frac{b - \mu}{\sigma}  \right] \\
&= \mathbb{P} \left[ \frac{a - \mu}{\sigma} \leq Z \leq \frac{b - \mu}{\sigma}  \right] \\
&= \mathbb{P} \left[ Z \leq \frac{b - \mu}{\sigma} \right] - \mathbb{P} \left[ Z \leq \frac{a - \mu}{\sigma} \right]
\end{align*}
So, we could calculate the probability that a normally distribution
variable $X$ is between two extremes as long as we had a table giving
the probability that $Z$ is less than any particular $\alpha$. A table
of such values is called a \textbf{Z-table}. These usually only include
values for a positive $\alpha$, as negative values can be inferred by
symmetry. Such a table is included in this handout and will given on
the final two exams.

Two particularly important scores are:
\begin{align*}
\mathbb{P}\left[ Z \leq 1.96 \right] \approx 0.9750 = 1 - 0.05/2 \\
\mathbb{P}\left[ Z \leq 2.56 \right] \approx 0.9948 = 1 - 0.01/2 \\
\end{align*}
Which yield what we refer to as tail bounds:
\begin{align*}
\mathbb{P}\left[ |Z| \geq 1.96 \right] \approx 0.05 \\
\mathbb{P}\left[ |Z| \leq 2.56 \right] \approx 0.01 \\
\end{align*}

\includegraphics[width=1.2\textwidth]{img/ztab.jpg}

\end{document}









