\documentclass{tufte-handout}

\usepackage{amssymb,amsmath}
% \usepackage{mathspec}
\usepackage{graphicx,grffile}
\usepackage{longtable}
\usepackage{booktabs}

\newtheorem{mydef}{Definition}
\newtheorem{thm}{Theorem}
\DeclareMathOperator*{\argmin}{arg\,min}
\DeclareMathOperator*{\argmax}{arg\,max}
\newcommand{\iid}{\stackrel{iid}{\sim}}

\providecommand{\tightlist}{%
  \setlength{\itemsep}{0pt}\setlength{\parskip}{0pt}}

\setlength{\parskip}{12pt}
\setlength\parindent{0pt}

\begin{document}

\justify

{\LARGE Handout 15: Method of Moments}

\vspace*{18pt}

\noindent

Today we explore a general procedures for constructing estimators
to estimate the parameters of a probability model from observed
data.

\section*{Method of Moments}

When the random sample one draws from a population is large, it is not
unreasonable to expect that the characteristics of the population will
be fairly well approximated by the characteristics of the sample. This
type of intuition gives us some confidence that the sample mean
$\bar{X}$
will be a reliable estimator of the population mean $\mu$ when the
sample size $n$ is large and, similarly, that the sample proportion
$p$ will be a reliable estimator of the population proportion
$p$ when $n$ is large. Taking advantage of the expected
similarity between a population and the sample drawn from it often lies
behind the parameter estimation strategy employed. The method of moments
is an approach to parameter estimation that is motivated by intuition
such as this. It is a method that produces, under
mild assumptions, consistent, asymptotically normal estimators of the
parameters of interest.

The moments of a probability distribution are, as we have discussed,
defined as:
\begin{align*}
\mu_k &= \mathbb{E} X^k
\end{align*}
The sample moments are defined similarly:
\begin{align*}
m_k &= \frac{1}{n} \sum_{i=1}^n X_i^k.
\end{align*}
The method of moments estimators simply solve the following system
of equations for the first $k$ moments:
\begin{align*}
\mu_1(\theta) &= m_1 \\
\mu_2(\theta) &= m_2 \\
&\vdots \\
\mu_k(\theta) &= m_k \\
\end{align*}
Where $k$ is set to the number of parameters in the model. They are often
easy to calculate both theoretically and numerically.

\end{document}









