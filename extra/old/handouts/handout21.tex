\documentclass{tufte-handout}

\usepackage{amssymb,amsmath}
\usepackage{mathtools}
\usepackage{graphicx,grffile}
\usepackage{longtable}
\usepackage{booktabs}

\newtheorem{mydef}{Definition}
\newtheorem{thm}{Theorem}

\providecommand{\tightlist}{%
  \setlength{\itemsep}{0pt}\setlength{\parskip}{0pt}}

\setlength{\parindent}{0em}
\setlength{\parskip}{12pt}

\begin{document}

\justify

{\LARGE Handout 21: Coins and Helicopters}

\vspace*{18pt}

\noindent
Each table has been given a bent coin and a paper helicopter.
Your goal is to estimate (a) the probability that the coin comes
up heads and (b) the time it takes the helicopter to hit the
ground once released. Starting with the coin, repeat all of the
steps below.

\textbf{1.} Describe a reasonable probability model to describe
the random process.

\textbf{2.} Calculate or look up from past worksheets the MME
for your probability model's parameters.

\textbf{3.} Calculate or look up from past worksheets the MLE
for your probability model's parameters.

\textbf{4.} Construct a reasonable prior distribution (hopefully
one that we know how to calculate the posterior from given your
likelihood). Decide on reasonable hyperparameters for the prior.
Calculate or look up from past worksheets the Bayes estimator.

\textbf{5.} Run 3 experiments and record the data. Make sure the
same person does the flipping or releasing to keep it consistent.

\textbf{6.} Compute all three estimators for the first 3 data points.

\textbf{7.} Compute a confidence interval for mean after the first
5 data points.

\textbf{8.} Run 17 (coin) / 7 (helicopter) additional random trials.

\textbf{9.} Update your three estimators and the confidence interval.

\textbf{10.} You should notice that the CI gets smaller and that the
three estimators are closer together after the second round of
experiments.

\end{document}









