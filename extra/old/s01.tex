\documentclass{beamer}

\title{Classical Inference I}
\author{Taylor Arnold}
\date{2024-01-01}

\usetheme{Madrid}
\usecolortheme{seahorse}

\newcommand{\Lim}[1]{\raisebox{0.5ex}{\scalebox{0.8}{$\displaystyle \lim_{#1}\;$}}}
\newcommand{\iid}{\overset{\mathrm{i.i.d.}}{\sim}}
\newcommand{\E}{\mathbb{E}}
\newcommand{\Prob}{\mathbb{P}}
\newcommand{\V}{\text{Var}}


\begin{document}

\frame{\titlepage}


%%%%%%%%%%%%%%%%%%%%%%%%%%%%%%%%%%%%%%%%%%%%%%%%%%%%%%%%%%%%%%%%%%%%%%%%%%%%%%%%%%%%%%%%
\begin{frame}
\frametitle{Random Sample}

We work with a random sample of size $n$, that is

\begin{align*}
X_1, \ldots, X_n &\iid G
\end{align*}

Where 

\begin{align*}
\E \left[ X_j \right] &= \mu, \\
\V \left[ X_j \right] &= \sigma^2.
\end{align*}

\end{frame}

%%%%%%%%%%%%%%%%%%%%%%%%%%%%%%%%%%%%%%%%%%%%%%%%%%%%%%%%%%%%%%%%%%%%%%%%%%%%%%%%%%%%%%%%
% set.seed(1L); x <- sort(round(rnorm(16, mean = 2, sd = 3), 1))
% clipr::write_clip(paste(x, collapse = ", "))
% mean(x)
\begin{frame}[fragile]
\frametitle{Potato Diet}

\begin{verbatim}
    -4.6, -0.5, -0.5, 0.1, 0.1, 1.1, 1.9, 2.6, 3,
          3.2, 3.5, 3.7, 4.2, 5.4, 6.5, 6.8
\end{verbatim}

\end{frame}


%%%%%%%%%%%%%%%%%%%%%%%%%%%%%%%%%%%%%%%%%%%%%%%%%%%%%%%%%%%%%%%%%%%%%%%%%%%%%%%%%%%%%%%%
\begin{frame}
\frametitle{Sample Mean}

\begin{align*}
\bar{X} &= \frac{1}{n} \times \left[X_1 + \cdots + X_n \right] \\
&= \frac{1}{n} \times \sum_i X_i \\
\end{align*}

\end{frame}


%%%%%%%%%%%%%%%%%%%%%%%%%%%%%%%%%%%%%%%%%%%%%%%%%%%%%%%%%%%%%%%%%%%%%%%%%%%%%%%%%%%%%%%%
\begin{frame}
\frametitle{Sample Mean: Expected Value}

Q1. What is the expected value of the sample mean? \pause

\begin{align*}
\E \bar{X} &= \E \left[ \frac{1}{n} \times \sum_i X_i \right] \\
&= \frac{1}{n} \times \E \left[ \sum_i X_i \right] \\
&= \frac{1}{n} \times \left[ \sum_i \E X_i \right] \\
&= \frac{1}{n} \times \left[ \sum_i \mu \right] \\
&= \frac{1}{n} \times n \cdot \mu = \mu.
\end{align*}

\end{frame}

%%%%%%%%%%%%%%%%%%%%%%%%%%%%%%%%%%%%%%%%%%%%%%%%%%%%%%%%%%%%%%%%%%%%%%%%%%%%%%%%%%%%%%%%
\begin{frame}
\frametitle{Sample Mean: Variance}

Q2. What is the variance of the sample mean? \pause

\begin{align*}
\V \bar{X} &= \V \left[ \frac{1}{n} \times \sum_i X_i \right] \\
&= \frac{1}{n^2} \times \V \left[ \sum_i X_i \right] \\
&= \frac{1}{n^2} \times \left[ \sum_i \V X_i \right] \\
&= \frac{1}{n^2} \times \left[ \sum_i \sigma^2 \right] \\
&= \frac{1}{n^2} \times n \cdot \sigma^2 = \frac{\sigma^2}{n}.
\end{align*}

\end{frame}

%%%%%%%%%%%%%%%%%%%%%%%%%%%%%%%%%%%%%%%%%%%%%%%%%%%%%%%%%%%%%%%%%%%%%%%%%%%%%%%%%%%%%%%%
\begin{frame}
\frametitle{Sample Mean: Distribution}

If $G$ is normal, then $\bar{X}$ is normal. Otherwise, we can
approximate the sample mean as a normal using the Central Limit Theorem (CLT).
So, let's assume that:
\begin{align*}
\bar{X} \sim N(\mu, \sigma^2 / n).
\end{align*}
\pause Consider $Z \sim N(0, 1)$. We (should) know that we can now write:
\begin{align*}
\bar{X} = \mu + Z \cdot \sqrt{\sigma^2 / n}.
\end{align*}

\end{frame}

%%%%%%%%%%%%%%%%%%%%%%%%%%%%%%%%%%%%%%%%%%%%%%%%%%%%%%%%%%%%%%%%%%%%%%%%%%%%%%%%%%%%%%%%
\begin{frame}
\frametitle{Standard Normal}

Recall that:
\begin{align*}
\Prob \left[ |Z| > 2.58 \right] \approx 0.01
\end{align*}

\end{frame}

%%%%%%%%%%%%%%%%%%%%%%%%%%%%%%%%%%%%%%%%%%%%%%%%%%%%%%%%%%%%%%%%%%%%%%%%%%%%%%%%%%%%%%%%
\begin{frame}
\frametitle{Standard Normal}

From which we can see that:
\begin{align*}
\Prob \left[ \left|\frac{\bar{X} - \mu}{\sqrt{\sigma^2 / n}} \right| > 2.58 \right] &\approx 0.01 \\
\Prob \left[ |\bar{X} - \mu| > 2.58 \cdot \sqrt{\sigma^2 / n} \right] &\approx 0.01
\end{align*}
Or, in other words, there is a 99\% chance that:
\begin{align*}
\bar{X} - 2.58 \cdot \sqrt{\sigma^2 / n} \leq \mu \leq \bar{X} + 2.58 \cdot \sqrt{\sigma^2 / n}
\end{align*}

\end{frame}

%%%%%%%%%%%%%%%%%%%%%%%%%%%%%%%%%%%%%%%%%%%%%%%%%%%%%%%%%%%%%%%%%%%%%%%%%%%%%%%%%%%%%%%%
\begin{frame}
\frametitle{Back to the Potato Diet}

The sample mean from our data is $\bar{x} = 2.28$ and $n = 16$.
\begin{align*}
2.28 - 2.58 \cdot \sqrt{3^2 / 16} \leq &\mu \leq 2.28 + 2.58 \cdot \sqrt{3^2 / 16} \\
2.28 - 2.58 \cdot 3/4 \leq &\mu \leq 2.28 + 2.58 \cdot 3/4 \\
0.345 \leq &\mu \leq 4.215.
\end{align*}

\end{frame}

\end{document}
