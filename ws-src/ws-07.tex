%%%%%%%%%%%%%%%%%%%%%%%%%%%%%%%%%%%%%%%%%%%%%%%%%%%%%%%%%%%%%%%%%%%%%%%%%%%%%%%%%
% QUESTION

Our goal is, broadly, to produce two independent random variabels that will have
chi-squared distributions under the null hypothesis. One will depend only on
the sample means and the other only on the sample variances. The latter is the
easier of the two, so let's start there. Consider the following quantity:
\begin{align*}
\sum_{j=1}^k \frac{(n_j - 1) S_j^2}{\sigma^2} &= \frac{(n_1 - 1) S_1^2}{\sigma^2} + \cdots + \frac{(n_j - 1) S_j^2}{\sigma^2}
\end{align*}
What is its distribution? Note that this result does not depend on the null
hypothesis

% SOLUTION

The $j$th term is, from our previous results, a $\chi^2(n_j - 1)$. Since all
of the terms are independent, the sum is just a $\chi^2(N - K)$, a chi-squared
with the sum of the degrees of freedom. This will be $N$ (the total number of
observations) minus $K$ (the number of blocks).

%%%%%%%%%%%%%%%%%%%%%%%%%%%%%%%%%%%%%%%%%%%%%%%%%%%%%%%%%%%%%%%%%%%%%%%%%%%%%%%%%
% QUESTION

Now, let's work out another chi-squared distribution based only on the means.
What is the distribution of the following quantity, where $\mu$ is the
hypothesized mean of all the blocks?
\begin{align*}
\left[\frac{\bar{X}_j - \mu}{\sqrt{\sigma^2 / n_j}}\right]^2
\end{align*} 

% SOLUTION

The part inside of the square is just a standardized sample mean. Squaring
in results in a $\chi^2(1)$.


%%%%%%%%%%%%%%%%%%%%%%%%%%%%%%%%%%%%%%%%%%%%%%%%%%%%%%%%%%%%%%%%%%%%%%%%%%%%%%%%%
% QUESTION

Next, what is this quantity?
\begin{align*}
\left[\frac{\bar{X} - \mu}{\sqrt{\sigma^2 / N}}\right]^2
\end{align*} 

% SOLUTION

By the same logic, this is also a $\chi^2(1)$.

%%%%%%%%%%%%%%%%%%%%%%%%%%%%%%%%%%%%%%%%%%%%%%%%%%%%%%%%%%%%%%%%%%%%%%%%%%%%%%%%%
% QUESTION

Using a similar derivation from worksheet 2, we can show that the following
is true:
\begin{align*}
\sum_{j=1}^K \left[\frac{\bar{X}_j - \mu}{\sqrt{\sigma^2 / n_j}}\right]^2 &=
\sum_{j=1}^K \left[\frac{\bar{X}_j - \bar{X}}{\sqrt{\sigma^2 / n_j}}\right]^2 +
\left[\frac{\bar{X} - \mu}{\sqrt{\sigma^2 / N}}\right]^2
\end{align*} 
Based on this and your previous results, what is the distribution of the 
second sum in the equation above?

% SOLUTION

The left-hand side is a sum of $K$ independent $\chi^2(1)$'s, which is a
$\chi^2(K)$. The last term is a single independent $\chi^2(1)$. So, because
chi-squared add together, the middle term must be a $\chi^2(K - 1)$.

%%%%%%%%%%%%%%%%%%%%%%%%%%%%%%%%%%%%%%%%%%%%%%%%%%%%%%%%%%%%%%%%%%%%%%%%%%%%%%%%%
% QUESTION

Put all of the previous results together to construct an F-statistic for the
hypothesis. Simplify the cancel the unknown value $\sigma^2$. While technically
either way is valid, put the chi-squared based on the means in the numerator and
the chi-squared based on the variances in the denominator. This will follow the
typical convention. Make sure to write down the distribution of the test statistic
under $H_0$.

% SOLUTION

We have the following:
\begin{align*}
F &= \frac{
  \frac{1}{K} \sum_{j=1}^K \left[\frac{\bar{X}_j - \bar{X}}{\sqrt{\sigma^2 / n_j}}\right]^2
}{
  \frac{1}{N-K} \sum_{j=1}^k \frac{(n_j - 1) S_1^j}{\sigma^2}
} = \frac{
  \frac{1}{K} \sum_{j=1}^K n_j \cdot \left[\bar{X}_j - \bar{X}\right]^2
}{
  \frac{1}{N-K} \sum_{j=1}^k (n_j - 1) S_1^j
} \sim F(K, N-K)
\end{align*}

