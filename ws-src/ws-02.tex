%%%%%%%%%%%%%%%%%%%%%%%%%%%%%%%%%%%%%%%%%%%%%%%%%%%%%%%%%%%%%%%%%%%%%%%%%%%%%%%%%
% QUESTION

Assume we have a random sample of size $n = 5$ with the following data:
$x_1 = -2$, $x_2 = -1$, $x_3 = 0$, $x_4 = 1$, $x_5 = 2$. What is the observered
sample variance $s^2$? 
\footnote{
  I tried to pick numbers to make the arithmetic easy to do. 
}

% SOLUTION

The mean of the data is clearly zero since the data are symetric around the
origin. So:
\begin{align*}
s^2 &= \frac{1}{5 - 1} \cdot \left[(-2)^2 + (-1)^2 + (0)^2 + (1)^2 + (2)^2 \right] \\
&= \frac{1}{4} \cdot \left[10 \right] = 5/2 = 2.5 \\
\end{align*}

%%%%%%%%%%%%%%%%%%%%%%%%%%%%%%%%%%%%%%%%%%%%%%%%%%%%%%%%%%%%%%%%%%%%%%%%%%%%%%%%%
% QUESTION

We are going to see two different ways to manipulate the formula for the 
sample variance. One is important for showing its general properties and
the other is important for showing the properties under normality. To start,
take the formula for $S^2_X$ and expand the square term. This will lead to
three different terms under the summation. Distribute the sum term and
try to simplify as much as possible. \textbf{Check with me to make sure
you have the full solution before continuing.}

% SOLUTION

We have the following based entirely on an algebraic manipulation of the
quantity, pulling out constants from the summation
\begin{align*}
S^2_X &= \frac{1}{n-1} \sum_{i=1}^n \left[X_i - \bar{X} \right]^2 \\
&= \frac{1}{n-1} \sum_{i=1}^n \left[X_i^2 + \bar{X}^2 - 2 X_i \bar{X} \right] \\
&= \frac{1}{n-1} \left[ \sum_{i=1}^n X_i^2 + \sum_{i=1}^n \bar{X}^2 - \sum_{i=1}^n 2 X_i \bar{X} \right]^2 \\
&= \frac{1}{n-1} \left[ \sum_{i=1}^n X_i^2 + n \cdot \bar{X}^2 - 2 \bar{X} \sum_{i=1}^n X_i \right]^2
\end{align*}
Now, from the definition of $\bar{X}$, we see that $\sum_i X_i = n \bar{X}$.
This gives:
\begin{align*}
S^2_X &= \frac{1}{n-1} \left[ \sum_{i=1}^n X_i^2 + n \cdot \bar{X}^2 - 2 \bar{X} \cdot n \cdot \bar{X} \right]^2 \\
&= \frac{1}{n-1} \left[ \sum_{i=1}^n X_i^2 - n \bar{X}^2 \right]^2
\end{align*}
That is the form that we need for the next step.

%%%%%%%%%%%%%%%%%%%%%%%%%%%%%%%%%%%%%%%%%%%%%%%%%%%%%%%%%%%%%%%%%%%%%%%%%%%%%%%%%
% QUESTION

What is the quantity $\mathbb{E}[X_i^2]$ in terms of the parameters $\mu_X$ and
$\sigma_X^2$? What is the quantity $\mathbb{E}[\bar{X}^2]$ Hint: You should be
able to use the variance formula $Var(Y) = \mathbb{E}Y^2 - [\mathbb{E}Y]^2$
to get this quickly from the setup of the problem.

% SOLUTION

Using the hint, we see that we have:
\begin{align*}
Var(X_i) &= \mathbb{E}X_i^2 - [\mathbb{E}X_i]^2 \\
\sigma_X^2 &= \mathbb{E}X_i^2 - \mu_X^2 \\
\mathbb{E}X_i^2 &= \sigma_X^2 - \mu_X^2.
\end{align*}
Similarly:
\begin{align*}
Var(\bar{X}) &= \mathbb{E}\bar{X}^2 - [\mathbb{E}\bar{X}]^2 \\
\frac{\sigma_X^2}{n}  &= \mathbb{E}\bar{X}^2 - \mu_X^2 \\
\mathbb{E}\bar{X}^2 &= \frac{\sigma_X^2}{n} - \mu_X^2.
\end{align*}
We will use these results in the next question.

%%%%%%%%%%%%%%%%%%%%%%%%%%%%%%%%%%%%%%%%%%%%%%%%%%%%%%%%%%%%%%%%%%%%%%%%%%%%%%%%%
% QUESTION

Compute $\mathbb{E}S_X^2$, starting with the results from the previous two
questions. What is the bias of $S_X^2$ as an estimator of $\sigma_X^2$?

% SOLUTION

Taking the expected value of quantity from before, and pushing the expectation
through all of the constants, we have:
\begin{align*}
\mathbb{E} S^2_X &= \frac{1}{n-1} \left[ \sum_{i=1}^n \mathbb{E} [X_i^2] - n \mathbb{E}[\bar{X}^2] \right]^2
\end{align*}
Plugging in the previous question and simplifying, we have:
\begin{align*}
\mathbb{E} S^2_X &= \frac{1}{n-1} \left[ \sum_{i=1}^n \left[ \sigma_X^2 - \mu_X^2 \right] -
  n \left[ \frac{\sigma_X^2}{n} - \mu_X^2 \right] \right]^2 \\
&= \frac{1}{n-1} \left[ n \cdot \left[ \sigma_X^2 - \mu_X^2 \right] -
  n \left[ \frac{\sigma_X^2}{n} - \mu_X^2 \right] \right]^2 \\
&= \frac{1}{n-1} \left[ n \sigma_X^2 - n \mu_X^2 - \sigma_X^2 + n \mu_X^2  \right]^2 \\
&= \frac{1}{n-1} \left[ (n-1) \sigma_X^2 \right]^2 \\
&= \sigma_X^2.
\end{align*}
Therefore, the statistic $S_X^2$ is an unbiased estimator of $\sigma_X^2$.

%%%%%%%%%%%%%%%%%%%%%%%%%%%%%%%%%%%%%%%%%%%%%%%%%%%%%%%%%%%%%%%%%%%%%%%%%%%%%%%%%
% QUESTION

Now, let's assume that $\mathcal{G}$ is a normal distribution. What is the
distribution of the following quantity from the sample mean $\bar{X}$ from a
random sample with $n$ observations?
\begin{align*}
\left[\frac{\bar{X} - \mu_X}{\sigma_X / \sqrt{n}} \right]^2.
\end{align*}
This should be a short answer based on what you derived last time.

% SOLUTION

The random variable inside of the square is a standard normal, so
the squared quantity has a $\chi^2(1)$ distribution.

%%%%%%%%%%%%%%%%%%%%%%%%%%%%%%%%%%%%%%%%%%%%%%%%%%%%%%%%%%%%%%%%%%%%%%%%%%%%%%%%%
% QUESTION

Now, let's consider the following quantity, which we will temporarily
give a name of $Y$ (it's not a quantity we need often, so there is not
a standard symbol for it):
\begin{align*}
Y &= \frac{1}{\sigma_X^2} \times \sum_i \left[ X_i - \mu_X \right]^2 
\end{align*}
What is the distribution of $Y$?

% SOLUTION

Moving the constant inside, we see that:
\begin{align*}
Y &=  \sum_i \left[ \frac{X_i - \mu_X}{\sigma_X} \right]^2 
\end{align*}
Each of the components of the sum have a distribution of $N(0, 1)$ and
every component is independent. So, $Y \sim \chi^2(n)$.

%%%%%%%%%%%%%%%%%%%%%%%%%%%%%%%%%%%%%%%%%%%%%%%%%%%%%%%%%%%%%%%%%%%%%%%%%%%%%%%%%
% QUESTION

The quantity $Y$ looks similar to $S_X^2$. We will use a common trick to
get $Y$ in terms of $S_X^2$: adding and subtracting the quantity $\bar{X}$
inside of the terms inside the sum. We can put the constant factor in later,
and so let's start with the following equality:
\begin{align*}
\sum_i \left[ X_i - \mu_X \right]^2 &= \sum_i \left[ X_i - \bar{X} + \bar{X} - \mu_X \right]^2  \\
&= \sum_i \left[ (X_i - \bar{X}) + (\bar{X} - \mu_X) \right]^2 
\end{align*}
Make sure that you see why this is valid! Starting with the
formula above, distribute the square. You should have three different summation
terms. Simplify by showing that the cross-term (the one with the $2$ in it) is
zero and another one of the terms is a constant in terms of the index $i$. The
third term should look similar to $S_X^2$. \textbf{Check with me to make sure
you have the full solution before continuing.}

% SOLUTION

We start by the straightforward
distribution of the squared term:
\begin{align*}
\sum_i \left[ X_i - \mu_X \right]^2 &= \sum_i \left[ (X_i - \bar{X}) + (\bar{X} - \mu_X) \right]^2 \\
&= \sum_i \left[ (X_i - \bar{X})^2 + (\bar{X} - \mu_X)^2 + 2 (X_i - \bar{X}) \cdot (\bar{X} - \mu_X) \right] \\
&= \sum_i (X_i - \bar{X})^2 + \sum_i (\bar{X} - \mu_X)^2 + \sum_i 2 (X_i - \bar{X}) \cdot (\bar{X} - \mu_X)
\end{align*}
Terms that do not have an $i$ index can come outside of the summation. The middle
term is all constant, so we just remove the sum by multiplying by $n$ (the number
of terms in the series):
\begin{align*}
\sum_i \left[ X_i - \mu_X \right]^2 
&= \sum_i (X_i - \bar{X})^2 + n \cdot (\bar{X} - \mu_X)^2 + 2 \cdot (\bar{X} - \mu_X) \cdot \sum_i  (X_i - \bar{X}) \\
&= \sum_i (X_i - \bar{X})^2 + n \cdot (\bar{X} - \mu_X)^2
\end{align*}
The last step comes from the fact that $\sum_i (X_i - \bar{X})$ must be zero. If
you do not believe that, distribute the summation and work out the details to see
why. 

%%%%%%%%%%%%%%%%%%%%%%%%%%%%%%%%%%%%%%%%%%%%%%%%%%%%%%%%%%%%%%%%%%%%%%%%%%%%%%%%%
% QUESTION

Divide both sides of your previous answer by $\sigma_X^2$. You
should have one term on the left and two on the right. Make one of the terms
on the right look like quantity in question 1.

% SOLUTION

\begin{align*}
\sum_i \left[ \frac{X_i - \mu_X}{\sigma_X} \right]^2 
&= \sum_i (X_i - \bar{X})^2 + n \cdot (\bar{X} - \mu_X)^2 + 2 \cdot (\bar{X} - \mu_X) \cdot \sum_i  (X_i - \bar{X}) \\
&= \frac{1}{\sigma_X^2} \sum_i (X_i - \bar{X})^2 + \frac{n}{\sigma_X^2} \cdot (\bar{X} - \mu_X)^2 \\
&= \frac{1}{\sigma_X^2} \sum_i (X_i - \bar{X})^2 + \left[\frac{\bar{X} - \mu_X}{\sigma_X / \sqrt{n}} \right]^2.
\end{align*}

%%%%%%%%%%%%%%%%%%%%%%%%%%%%%%%%%%%%%%%%%%%%%%%%%%%%%%%%%%%%%%%%%%%%%%%%%%%%%%%%%
% QUESTION

Using the previous set of results, what is the distribution of
the following quantity?
\begin{align*}
\frac{1}{\sigma_X^2} \sum_i (X_i - \bar{X})^2 &= \frac{(n-1) S_X^2}{\sigma_X^2}
\end{align*}

% SOLUTION

The left-hand side of the previous answer is $\chi^2(n)$ and the right-hand
side is the sum of two independent terms: a $\chi^2(1)$ and the value above. Therefore,
the term above must be a $\chi^2(n-1)$ (because then there sum would be a $\chi^2(n)$,
as required). 

%%%%%%%%%%%%%%%%%%%%%%%%%%%%%%%%%%%%%%%%%%%%%%%%%%%%%%%%%%%%%%%%%%%%%%%%%%%%%%%%%
% QUESTION

From probability theory, we have that the expected value of a random variable with
a chi-squared distribution with $k$ degrees of freedom is $k$. Its variance is $2k$.
Take the expected value of the quantity from the previous question and
simplify to get the expected value of $S_X^2$. You should see again that $S_X^2$ is
an unbiased estimator of $\sigma_X^2$, which we already proved in the general case
above.

% SOLUTION

We have:
\begin{align*}
\E \left[ \frac{(n-1) S_X^2}{\sigma_X^2} \right] &= n-1 \\
\frac{(n-1)}{\sigma_X^2} \cdot \E \left[ S_X^2 \right] &= n-1 \\
\E \left[ S_X^2 \right] &= \sigma_X^2
\end{align*}
So the expected value of the sample variance is equal to the population variance.
We say that this is an unbiased estimator of the variance, a concept that we will
return to in the next unit.

%%%%%%%%%%%%%%%%%%%%%%%%%%%%%%%%%%%%%%%%%%%%%%%%%%%%%%%%%%%%%%%%%%%%%%%%%%%%%%%%%
% QUESTION

Take the variance of the quantity you started with in the previous question and
simplify to get the variance of $S_X^2$.\footnote{
  I won't make you go through the steps again, but you should be able to see
  quickly from the result for $|bar{X}$ that any unbiased estimator that has
  a variance that limits to zero as $n\rightarrow\infty$ will be consistent
  by the application of Chebyshev's Inequality.
}

% SOLUTION

This is relatively straightforward based on the chi-squared distribution:
\begin{align*}
\V \left[ \frac{(n-1) S_X^2}{\sigma_X^2} \right] &= 2(n-1) \\
\frac{(n-1)^2}{\sigma_X^4} \cdot \V \left[ S_X^2 \right] &= 2(n-1) \\
\V \left[ S_X^2 \right] &= \frac{\sigma_X^4}{n-1}
\end{align*}

