%%%%%%%%%%%%%%%%%%%%%%%%%%%%%%%%%%%%%%%%%%%%%%%%%%%%%%%%%%%%%%%%%%%%%%%%%%%%%%%%%
% QUESTION

Consider a simple linear regression where the first $n/2$ values of $x_i$ are 
zero and the second $n/2$ values of $x_i$ are 1. We can use this to model the
mean of a variable that depends on whether the corresponding $x_i$ is in group
0 or group 1. Specifically, how do the means of $Y_i$ in these two groups
correspond to the parameters $b_0$ and $b_1$?

% SOLUTION

The parameter $b_0$ is the mean of $Y_i$ in group 0 and the parameter $b_1$ gives
the increase of the mean in group 1. So, group 1 has mean $b_0 + b_1$. Notice that
checking the hypothesis $H_0: b_1 = 0$ is the same as checking where there is no
difference in mean across the two groups.

%%%%%%%%%%%%%%%%%%%%%%%%%%%%%%%%%%%%%%%%%%%%%%%%%%%%%%%%%%%%%%%%%%%%%%%%%%%%%%%%%
% QUESTION

Let $\bar{y}_A$ be the mean of the first $n/2$ values of $Y_i$ and let $\bar{y}_B$
be the mean of the second $n/2$ values of $Y_i$. Consider the following form of the
MLE for $\widehat{b}_1$:\footnote{
  It can be shown that it is equivalent to the form on Worksheet 14.
}
\begin{align*}
\widehat{b}_1 &= \frac{\sum_i (y_i - \bar{y}) (x_i)}{\sum_i (x_i - \bar{x})^2}.
\end{align*}
Find a simple formula for $\widehat{b}_1$ in terms of $\bar{y}_A$ and $\bar{y}_B$.

% SOLUTION

Note that $\bar{y} = \frac{1}{2} \cdot (\bar{y}_A + \bar{y}_B)$, an important
identity that we will use below. Then, plugging in the definition, we have:
\begin{align*}
\widehat{b}_1 &= \frac{\sum_i (y_i - \bar{y}) (x_i)}{\sum_i (x_i - \bar{x})^2} \\
&= \frac{\sum_{i=n/2+1}^{n} (y_i - \bar{y})}{n/4} \\
&= \frac{4}{n} \times \sum_{i=n/2+1}^{n} (y_i - \bar{y}) \\
&= \frac{4}{n} \times \left[ \frac{n}{2} \bar{y}_b - \frac{n}{2} \bar{y} \right] \\
&= 2 \left[ \bar{y}_b - \bar{y} \right] \\
&= 2 \left[ \bar{y}_b - \frac{1}{2} \cdot (\bar{y}_A + \bar{y}_B) \right] \\
&= \bar{y}_B - \bar{y}_A.
\end{align*}

%%%%%%%%%%%%%%%%%%%%%%%%%%%%%%%%%%%%%%%%%%%%%%%%%%%%%%%%%%%%%%%%%%%%%%%%%%%%%%%%%
% QUESTION

Continuing from the previous question, find a simple formula for $\widehat{b}_0$
in terms of $\bar{y}_A$ and $\bar{y}_B$.

% SOLUTION

Here we have, by plugging from the previous equation, the following:
\begin{align*}
\widehat{b}_0 &= \bar{y} - (\bar{y}_B - \bar{y}_A) \cdot \bar{x} \\
&= \frac{1}{2} \cdot (\bar{y}_A + \bar{y}_B) - (\bar{y}_B - \bar{y}_A) \cdot \frac{1}{2} \\
&= \frac{1}{2} \cdot (\bar{y}_A + \bar{y}_B - \bar{y}_B - \bar{y}_A) \\
&= \frac{1}{2} \cdot (2 \bar{y}_A) \\
&= \bar{y}_A
\end{align*}

%%%%%%%%%%%%%%%%%%%%%%%%%%%%%%%%%%%%%%%%%%%%%%%%%%%%%%%%%%%%%%%%%%%%%%%%%%%%%%%%%
% QUESTION

What is the connection between the linear regression here and a two-sample T-test
for the means across the two groups?

% SOLUTION

They are exactly equivalent techniques! Just one reason that some people claim
regressions are really all that you need.