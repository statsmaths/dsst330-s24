%%%%%%%%%%%%%%%%%%%%%%%%%%%%%%%%%%%%%%%%%%%%%%%%%%%%%%%%%%%%%%%%%%%%%%%%%%%%%%%%%
% QUESTION

Assume we have a random sample of size $n = 5$ with the following data:
$x_1 = 2$, $x_1 = 6$, $x_1 = 1$, $x_1 = 0$, $x_1 = 6$. What is the observered
sample mean $\bar{x}$? 
\footnote{
  I am using the standard convention that we replace upper-case random 
  variable names with lower-case variables when we have specific 
  observations of them. 
}

% SOLUTION

We have:
\begin{align*}
\bar{x} &= \frac{2 + 6 + 1 + 0 + 6}{5} = \frac{15}{5} = 3.
\end{align*}

%%%%%%%%%%%%%%%%%%%%%%%%%%%%%%%%%%%%%%%%%%%%%%%%%%%%%%%%%%%%%%%%%%%%%%%%%%%%%%%%%
% QUESTION

Let $X_1, \ldots, X_n \iid \mathcal{G}$ be a random sample from a distribution
with mean $\mu_X$ and variance $\sigma^2_X$. What is the expected value of the
sample mean $\bar{X}$?\footnote{
  I gave the answer on the handout. Make sure that you can justify
  the result.
} Does this imply that $\bar{X}$ is an unbiased estimator of $\mu_X$?

% SOLUTION

We have:
\begin{align*}
\E \bar{X} &= \E \left[ \frac{1}{n} \times \sum_{i=1}^n X_i \right] \\
&= \frac{1}{n} \times \E \left[ \sum_{i=1}^n X_i \right] \\
&= \frac{1}{n} \times \left[ \sum_{i=1}^n \E X_i \right] \\
&= \frac{1}{n} \times \left[ \sum_{i=1}^n \mu_X \right] \\
&= \frac{1}{n} \times n \cdot \mu_X = \mu_X.
\end{align*}
So by the definition of unbiased, $\bar{X}$ is an unbiased estimator of $\mu_X$.


%%%%%%%%%%%%%%%%%%%%%%%%%%%%%%%%%%%%%%%%%%%%%%%%%%%%%%%%%%%%%%%%%%%%%%%%%%%%%%%%%
% QUESTION

Using the same set-up as the previous question, what is $Var(\bar{X})$?

% SOLUTION

We have:
\begin{align*}
\V [\bar{X}] &= \V \left[ \frac{1}{n} \times \sum_{i=1}^n X_i \right] \\
&= \frac{1}{n^2} \times \V \left[ \sum_{i=1}^n X_i \right] \\
&= \frac{1}{n^2} \times \left[ \sum_{i=1}^n \V X_i \right] \\
&= \frac{1}{n^2} \times \left[ \sum_{i=1}^n \sigma_X^2 \right] \\
&= \frac{1}{n^2} \times n \cdot \sigma_X^2 = \frac{\sigma_X^2}{n}.
\end{align*}
As given on the handout.

%%%%%%%%%%%%%%%%%%%%%%%%%%%%%%%%%%%%%%%%%%%%%%%%%%%%%%%%%%%%%%%%%%%%%%%%%%%%%%%%%
% QUESTION

Let $Y$ be a random variable with mean $m$ and variance $v$. Chebyshev's
Inequality tells us that if for any $a > 0$,
\begin{align*}
\mathbb{P}[ |Y - m| \geq a] \leq \frac{v}{a^2}.
\end{align*}
Use this result to show that $\bar{X}$ is a consistent estimator of $\mu_X$.

% SOLUTION

Apply Chebyshev's inequality with $Y = \bar{X}$ and $a = \epsilon$, to get
that for any $\epsilon$ we have (using the two previous results):
\begin{align*}
\mathbb{P}[ |\bar{X} - \mu_X| \geq \epsilon] \leq \frac{\sigma_X^2}{\epsilon^2 n}.
\end{align*}
Since $\epsilon$ and $\sigma_X^2$ are assumed to be fixed, we have that:
\begin{align*}
\lim_{n\rightarrow \infty} \mathbb{P}[ |\bar{X} - \mu_X| \geq \epsilon] = 0.
\end{align*}
By definition, then, $\bar{X}$ is a consistent estimator of $\mu_X$.

%%%%%%%%%%%%%%%%%%%%%%%%%%%%%%%%%%%%%%%%%%%%%%%%%%%%%%%%%%%%%%%%%%%%%%%%%%%%%%%%%
% QUESTION

Assume that $\mathcal{G}$ has a normal distribution. Define the following:
\begin{align*}
Z &= \frac{\mu_X - \bar{X}}{\sqrt{\sigma_X^2 / n}}
\end{align*}
What is the distribution of Z?

% SOLUTION

We see that $Z$ is a scaled version of a normal distribution, and therefore
$Z$ must also be normal. All that is left is to determine its mean and variance.
These are:
\begin{align*}
\E Z &= \E \left[ \frac{\mu_X - \bar{X}}{\sqrt{\sigma_X^2 / n}} \right] \\
&= \frac{\mu_X - \E\bar{X}}{\sqrt{\sigma_X^2 / n}} \\
&= \frac{\mu_X - \mu_X}{\sqrt{\sigma_X^2 / n}} \\
&= 0.
\end{align*}
And 
\begin{align*}
\V Z &= \V \left[ \frac{\mu_X - \bar{X}}{\sqrt{\sigma_X^2 / n}} \right] \\
&= \V \left[ \frac{\bar{X}}{\sqrt{\sigma_X^2 / n}} \right] \\
&= \frac{1}{\sqrt{\sigma_X^2 / n}} \times \V \bar{X} \\
&= \frac{\frac{\sigma_X^2}{n}}{\sqrt{\sigma_X^2 / n}} \\
&= 1.
\end{align*}
So, $Z \sim N(0, 1)$, a standard normal.