%%%%%%%%%%%%%%%%%%%%%%%%%%%%%%%%%%%%%%%%%%%%%%%%%%%%%%%%%%%%%%%%%%%%%%%%%%%%%%%%%
% QUESTION

Let $X \sim N(\mu, \sigma^2)$, with $\sigma^2 > 0$ a fixed and known constant.
(a) Compute the Fisher Information $\mathcal{I}(\mu)$. (b) The MLE for $\mu$
is equal to $X$ (generally it's the mean, but in the one-observation case the
mean is equal to $X$). Find the efficency of the MLE.

% SOLUTION

(a) We have the following for the first derivative the of the log likelihood:
\begin{align*}
\frac{\partial}{\partial \mu} \log(f(\mu; x))
&= \frac{\partial}{\partial \mu} \left[ \frac{-1}{2\sigma^2} (x - \mu)^2 \right] \\
&= \frac{+2}{2\sigma^2} (x - \mu) \\
&= \frac{1}{\sigma^2} (x - \mu)
\end{align*}
And for the second derivative:
\begin{align*}
\frac{\partial^2}{\partial^2 \mu} \log(f(\mu; x))
&= \frac{\partial}{\partial \mu} \left[ \frac{1}{\sigma^2} (x - \mu) \right] \\
&= \frac{-1}{\sigma^2}.
\end{align*}
Then, the Fisher information is:
\begin{align*}
\mathcal{I}(\mu) &= - \mathbb{E} \left[ \frac{\partial^2}{\partial \mu^2} \log f(\mu; x) \right] \\
&= \mathbb{E} \left[ \frac{1}{\sigma^2} \right] \\
&= \frac{1}{\sigma^2}.
\end{align*}
(b) The variance of the MLE is equal to:
\begin{align*}
Var(\hat{\mu}) &= Var(X) = \sigma^2.
\end{align*}
And therefore the efficency is:
\begin{align*}
e(\hat{\mu}) &= \frac{\mathcal{I}(\theta)^{-1}}{Var(\widehat{\theta})} = \frac{\sigma^2}{\sigma^2} = 1.
\end{align*}
So, the MLE is optimally efficent. It does as well as any unbiased estimator can do in
terms of predicting the value of $\mu$ from the data.

%%%%%%%%%%%%%%%%%%%%%%%%%%%%%%%%%%%%%%%%%%%%%%%%%%%%%%%%%%%%%%%%%%%%%%%%%%%%%%%%%
% QUESTION

Let $X \sim Poisson(\lambda)$. (a) Compute the Fisher Information $\mathcal{I}(\lambda)$.
(b) The MLE for $\lambda$ is equal to $X$ (generally it's the mean, but in the one-observation
case the mean is equal to $X$). Find the efficency of the MLE.

% SOLUTION

(a) We have the following for the first derivative the of the log likelihood:
\begin{align*}
\frac{\partial}{\partial \lambda} \log(f(\lambda; x))
&= \frac{\partial}{\partial \lambda} \left[ x \log(\lambda) - \lambda + \log(x!) \right] \\
&= \frac{x}{\lambda} - 1.
\end{align*}
And for the second derivative:
\begin{align*}
\frac{\partial^2}{\partial^2 \lambda} \log(f(\lambda; x))
&= \frac{\partial}{\partial \lambda} \left[ \frac{x}{\lambda} - 1 \right] \\
&= \frac{-x}{\lambda^2}.
\end{align*}
Then, the Fisher information is:
\begin{align*}
\mathcal{I}(\lambda) &= - \mathbb{E} \left[ \frac{\partial^2}{\partial \lambda^2} \log f(\lambda; x) \right] \\
&= \mathbb{E} \left[ \frac{x}{\lambda^2} \right] \\
&= \frac{\lambda}{\lambda^2} = \frac{1}{\lambda}.
\end{align*}
(b) The variance of the MLE is equal to:
\begin{align*}
Var(\hat{\lambda}) &= Var(X) = \lambda.
\end{align*}
And therefore the efficency is:
\begin{align*}
e(\hat{\lambda}) &= \frac{\mathcal{I}(\theta)^{-1}}{Var(\widehat{\theta})} = \frac{\lambda}{\lambda} = 1.
\end{align*}
So, the MLE is optimally efficent. It does as well as any unbiased estimator can do in
terms of predicting the value of $\lambda$ from the data.

%%%%%%%%%%%%%%%%%%%%%%%%%%%%%%%%%%%%%%%%%%%%%%%%%%%%%%%%%%%%%%%%%%%%%%%%%%%%%%%%%
% QUESTION

Let $X \sim Binomial(n, p)$ with $n>0$ a fixed and known constant.
(a) Compute the Fisher Information $\mathcal{I}(p)$.\footnote{
  Try to simplify this as much as possible. You should be able to
  get something that has a denominator equal to $p(1-p)$.
} (b) The MLE for $p$ is
equal to $X/n$. Find the efficency of the MLE.

% SOLUTION

(a) We have the following for the first derivative the of the log likelihood:
\begin{align*}
\frac{\partial}{\partial p} \log(f(p; x))
&= \frac{\partial}{\partial p} \left[ \log(\binom{n}{x}) +  x \cdot \log(p) + (n-x) \cdot \log(1-p) \right] \\
&= \frac{x}{p} - \frac{n-x}{1-p}.
\end{align*}
And for the second derivative:
\begin{align*}
\frac{\partial^2}{\partial^2 p} \log(f(p; x))
&= \frac{\partial}{\partial p} \left[ \frac{x}{p} - \frac{n-x}{1-p} \right] \\
&= \frac{-x}{p^2} - \frac{n-x}{(1-p)^2}
\end{align*}
Then, the Fisher information is:
\begin{align*}
\mathcal{I}(p) &= - \mathbb{E} \left[ \frac{\partial^2}{\partial p^2} \log f(p; x) \right] \\
&= \mathbb{E} \left[ \frac{x}{p^2} + \frac{n-x}{(1-p)^2} \right] \\
&= \frac{np}{p^2} + \frac{n-np}{(1-p)^2} \\
&= \frac{n}{p} + \frac{n(1-p)}{(1-p)^2} \\
&= \frac{n}{p} + \frac{n}{(1-p)} \\
&= n \cdot \left[ \frac{1}{p} + \frac{1}{1-p} \right] \\
&= n \cdot \left[ \frac{(1 - p) + p}{p (1 - p)} \right] \\
&= n \cdot \left[ \frac{1}{p (1 - p)} \right] \\
&= \frac{n}{p (1 - p)}
\end{align*}
(b) The variance of the MLE is equal to:
\begin{align*}
Var(\hat{p}) &= Var(X/n) = \frac{np(1-p)}{n^2} = \frac{p(1-p)}{n}.
\end{align*}
And therefore the efficency is:
\begin{align*}
e(\hat{p}) &= \frac{\mathcal{I}(\theta)^{-1}}{Var(\widehat{\theta})} = \frac{\frac{p (1 - p)}{n}}{\frac{p(1-p)}{n}} = 1.
\end{align*}
So, the MLE is optimally efficent. It does as well as any unbiased estimator can do in
terms of predicting the value of $p$ from the data.






