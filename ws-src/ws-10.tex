%%%%%%%%%%%%%%%%%%%%%%%%%%%%%%%%%%%%%%%%%%%%%%%%%%%%%%%%%%%%%%%%%%%%%%%%%%%%%%%%%
% QUESTION

\textbf{(Ratio Test)} Let $X_1, \ldots, X_n \iid Exp(\lambda)$.
What is the test statistic $\Lambda$ for the corresponding likelihood 
ratio test for the null hypothesis $H_0: \lambda = 1$.

% SOLUTION

From last time, we know that the MLE will be $\bar{x}^{-1}$. So, the likelihood
ratio test will be:
\begin{align*}
\Lambda &= -2 \cdot \sum_i \left[ l(1, x_i) - l(\bar{x}^{-1}, x_i) \right] \\
&= -2 \cdot \sum_i \left[ log(1) + (-1 \cdot x_i) - \log(x^{-1}) - (-1 x^{-1} \cdot x_i) \right] \\
&= 2 \cdot \sum_i \left[ x_i + \log(\bar{x}^{-1}) - \bar{x}^{-1} \cdot x_i) \right] \\
&= 2 \cdot \left[ n \bar{x} + n \log(\bar{x}^{-1}) - n \right] \\
&= 2n \cdot \left[ \bar{x} + \log(\bar{x}^{-1}) - 1 \right].
\end{align*}

%%%%%%%%%%%%%%%%%%%%%%%%%%%%%%%%%%%%%%%%%%%%%%%%%%%%%%%%%%%%%%%%%%%%%%%%%%%%%%%%%
% QUESTION

\textbf{(Ratio Test)} Let $X_1, \ldots, X_n \iid Poisson(\lambda)$.
What is the test statistic $\Lambda$ for the corresponding likelihood 
ratio test for the null hypothesis $H_0: \lambda = 1$.

% SOLUTION

TODO

%%%%%%%%%%%%%%%%%%%%%%%%%%%%%%%%%%%%%%%%%%%%%%%%%%%%%%%%%%%%%%%%%%%%%%%%%%%%%%%%%
% QUESTION

\textbf{(Ratio Test)} Let $X_1, \ldots, X_n \iid Bernoulli(p)$.
What is the test statistic $\Lambda$ for the corresponding likelihood 
ratio test for the null hypothesis $H_0: p = 0.2$.

% SOLUTION

TODO

%%%%%%%%%%%%%%%%%%%%%%%%%%%%%%%%%%%%%%%%%%%%%%%%%%%%%%%%%%%%%%%%%%%%%%%%%%%%%%%%%
% QUESTION

\textbf{(Ratio Test)} Let $X \sim Bin(n, p_1)$ and $Y \sim Bin(n, p_2)$ be
independent random variables. We want to test the hypothesis that
$H_0: p_1 = p_2$. What are the corresponding $\Theta$ and $\Theta_0$ in
our updated formulation of hypothesis testing?\footnote{
  We will derive the actual test itself in a more general form next
  class.
} If we use a Likelihood Ratio Test for this hypothesis, how many degrees of
freedom should $\Lambda$ have?

% SOLUTION

TODO

%%%%%%%%%%%%%%%%%%%%%%%%%%%%%%%%%%%%%%%%%%%%%%%%%%%%%%%%%%%%%%%%%%%%%%%%%%%%%%%%%
% QUESTION

\textbf{(Ratio Test)} Recall that we used the one-sample ANOVA test with the
null-hypothesis that the means of $K$ samples are all the same. Write down and
describe the values of $\Theta$ and $\Theta_0$ that correspond to this test. 
If we use a Likelihood Ratio Test for this hypothesis, how many degrees of
freedom should $\Lambda$ have?

% SOLUTION

TODO

%%%%%%%%%%%%%%%%%%%%%%%%%%%%%%%%%%%%%%%%%%%%%%%%%%%%%%%%%%%%%%%%%%%%%%%%%%%%%%%%%
% QUESTION

\textbf{(MLE Practice)} Let $X_1, \ldots, X_n \iid Uniform(0, a)$. Find the MLE
estimator for $a$. Note: You cannot do this using the derivative. Just think about it!

% SOLUTION

TODO
