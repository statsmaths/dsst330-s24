%%%%%%%%%%%%%%%%%%%%%%%%%%%%%%%%%%%%%%%%%%%%%%%%%%%%%%%%%%%%%%%%%%%%%%%%%%%%%%%%%
% QUESTION

\textbf{(Ratio Test)} Let $X_1, \ldots, X_n \iid Exp(\lambda)$.
What is the test statistic $\Lambda$ for the corresponding likelihood 
ratio test for the null hypothesis $H_0: \lambda = 1$.

% SOLUTION

In general for the exponential distribution the likelihood ratio test statistic
will be:
\begin{align*}
\Lambda &= 2 \cdot \sum_i \left[ l(\hat{\lambda}) - l(\lambda_0)  \right] \\
&= 2 \cdot \sum_i \left[ \log(\hat{\lambda}) - \hat{\lambda} x_i - \log(\lambda_0) + \lambda_0 x_i  \right] \\
&= 2 \cdot \left[ \sum_i \log(\hat{\lambda}) - \hat{\lambda} \sum_i x_i - \sum_i \log(\lambda_0) + \sum_i \lambda_0 x_i  \right] \\
&= 2 \cdot \left[ n \log(\hat{\lambda}) - \hat{\lambda} n \cdot \bar{x} - n \log(\lambda_0) + \lambda_0 \cdot n \bar{x} \right] \\
&= 2 \cdot n \left[ \log(\hat{\lambda}) - \hat{\lambda} \bar{x} - \log(\lambda_0) + \lambda_0 \bar{x} \right]
\end{align*}
We know from last time that $\hat{\lambda} = \bar{x}^{-1}$ and have that $\lambda_0 = 1$
from the null-hypothesis. So:
\begin{align*}
\Lambda &= 2 \cdot n \left[ \log(\bar{x}^{-1}) - \bar{x}^{-1} \bar{x} - \log(1) + 1 \cdot \bar{x} \right] \\
&= 2 \cdot n \left[ - \log(\bar{x}) - 1 - 0 + \bar{x} \right] \\
&= 2 \cdot n \left[ \bar{x} - \log(\bar{x}) - 1  \right].
\end{align*}
And that's as much as we can simplify.

%%%%%%%%%%%%%%%%%%%%%%%%%%%%%%%%%%%%%%%%%%%%%%%%%%%%%%%%%%%%%%%%%%%%%%%%%%%%%%%%%
% QUESTION

\textbf{(Ratio Test)} Let $X_1, \ldots, X_n \iid Poisson(\lambda)$.
What is the test statistic $\Lambda$ for the corresponding likelihood 
ratio test for the null hypothesis $H_0: \lambda = 1$.

% SOLUTION

In general for the Poisson distribution, the likelihood ratio test statistic
will be:
\begin{align*}
\Lambda &= 2 \cdot \sum_i \left[ l(\hat{\lambda}) - l(\lambda_0)  \right] \\
&= 2 \cdot \sum_i \left[ x_i \log(\hat{\lambda}) - \hat{\lambda} - \log(x_i!) - 
                         x_i \log(\lambda_0) + \lambda_0 + \log(x_i!)  \right] \\
&= 2 \cdot n \left[ \bar{x} \log(\hat{\lambda}) - \hat{\lambda} - \bar{x} \log(\lambda_0) + \lambda_0 \right] 
\end{align*}
We know that $\hat{\lambda} = \bar{x}$ for a Poisson MLE and have $\lambda_0 = 1$
from the null-hypothesis, so:
\begin{align*}
\Lambda
&= 2 \cdot n \sum_i \left[ \bar{x} \log(\bar{x}) - \bar{x} - \bar{x} \log(1) + \lambda_0 \right] \\ 
&= 2 \cdot n \sum_i \left[ \bar{x} \log(\bar{x}) - \bar{x} + \lambda_0 \right] \\ 
\end{align*}
And that's about all that we can do. 

%%%%%%%%%%%%%%%%%%%%%%%%%%%%%%%%%%%%%%%%%%%%%%%%%%%%%%%%%%%%%%%%%%%%%%%%%%%%%%%%%
% QUESTION

\textbf{(Ratio Test)} Let $X_1, \ldots, X_n \iid Bernoulli(p)$.
What is the test statistic $\Lambda$ for the corresponding likelihood 
ratio test for the null hypothesis $H_0: p = 0.2$.

% SOLUTION

One more, same idea. In general for the Bernouilli distribution, the likelihood
ratio test statistic will be:
\begin{align*}
\Lambda
&= 2 \cdot \sum_i \left[ l(\hat{p}) - l(p_0)  \right] \\
&= 2 \cdot \sum_i \left[ x_i \log(\hat{p}) + (1-x_i) \log(1 - \hat{p}) -
                         x_i \log(p_0) - (1-x_i) \log(1 - p_0) \right] \\
&= 2 \cdot n \left[ \bar{x} \log(\hat{p}) + (1-\bar{x}) \log(1 - \hat{p}) -
                    \bar{x} \log(p_0) - (1-\bar{x}) \log(1 - p_0) \right]                       
\end{align*}
Plugging in the MLE ($\hat{p} = \bar{x}$) and the null hypothesis ($p_0 = 0.2$),
we have:
\begin{align*}
\Lambda
&= 2 \cdot n \left[ \bar{x} \log(\bar{x}) + (1-\bar{x}) \log(1 - \bar{x}) -
                    \bar{x} \log(0.2) - (1-\bar{x}) \log(1 - 0.2) \right]                       
\end{align*}
And again, there's not much more to simplify here. We could easily compute
the value of $\Lambda$ for a particular dataset, and then compare the a 
chi-squared distribution.

%%%%%%%%%%%%%%%%%%%%%%%%%%%%%%%%%%%%%%%%%%%%%%%%%%%%%%%%%%%%%%%%%%%%%%%%%%%%%%%%%
% QUESTION

\textbf{(Ratio Test)} Let $X \sim Bin(n, p_1)$ and $Y \sim Bin(n, p_2)$ be
independent random variables, assuming that $n$ is a known quantity. We want
to test the hypothesis that $H_0: p_1 = p_2$. What are the corresponding
$\Theta$ and $\Theta_0$ in
our updated formulation of hypothesis testing?\footnote{
  We will derive the actual test itself in a more general form next
  class.
} If we use a Likelihood Ratio Test for this hypothesis, how many degrees of
freedom should $\Lambda$ have?

% SOLUTION

We will write values of the parameter $\theta = (p_1, p_2)$. These can be
any values between $0$ and $1$, so  $\Theta = [0, 1] \times [0, 1] \subset \mathbb{R}^2$.
The null-hypothesis is then the subset of this where the two values are equal:
$\Theta_0 = \{(x, y) | x = y, x \in [0, 1] \} \subset \Theta$. Visually, this 
is a line of values between the origin and the point $(1, 1)$. The difference
in dimensionality is $2 - 1 = 1$, so $\Lambda \sim \chi^2(1)$.

%%%%%%%%%%%%%%%%%%%%%%%%%%%%%%%%%%%%%%%%%%%%%%%%%%%%%%%%%%%%%%%%%%%%%%%%%%%%%%%%%
% QUESTION

\textbf{(Ratio Test)} Recall that we used the one-sample ANOVA test with the
null-hypothesis that the means of $K$ samples are all the same. Write down and
describe the values of $\Theta$ and $\Theta_0$ that correspond to this test. 
If we use a Likelihood Ratio Test for this hypothesis, how many degrees of
freedom should $\Lambda$ have?

% SOLUTION

Here, we will let $\theta = (\mu_1, \ldots, \mu_k, \sigma^2) \in \mathbb{R}^{k+1}$.
For the parameter space we have:
\begin{align*}
\Theta &= \mathbb{R}^k \times (0, \infty) \\
&= \{ (x_1, \ldots, x_k, x_{k+1}) | x_{k+1} > 0 \}
\end{align*}
The null hypothesis is the set where the first k elements are equal:
\begin{align*}
\Theta_0 &= \{ (x_1, \ldots, x_k, x_{k+1}) | x_{i} = x_{j} (i <= k, j <=k), x_{k+1} > 0 \}
\end{align*}

%%%%%%%%%%%%%%%%%%%%%%%%%%%%%%%%%%%%%%%%%%%%%%%%%%%%%%%%%%%%%%%%%%%%%%%%%%%%%%%%%
% QUESTION

\textbf{(MLE Practice)} Let $X_1, \ldots, X_n \iid Uniform(0, a)$. Find the MLE
estimator for $a$. Note: You cannot do this using the derivative. Just think about it!

% SOLUTION

The density $f(x)$ of the uniform distribution from $0$ to $a$ will be $1/a$ if 
$x \in [0, a]$ and zero otherwise. So, to maximize the likelihood, clearly we need
to have $a \leq max_i \{x_i\}$ (otherwise the likelihood is zero). But, as long as the
maximum is no larger than $a$, the likelihood decreases with with a larger $a$. So,
$\hat{a} = max_i \{x_i\}$. If that's not clear, ask me to draw you a picture in class!
