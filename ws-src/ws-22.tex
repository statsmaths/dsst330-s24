%%%%%%%%%%%%%%%%%%%%%%%%%%%%%%%%%%%%%%%%%%%%%%%%%%%%%%%%%%%%%%%%%%%%%%%%%%%%%%%%%
% QUESTION

Consider running four hypothesis tests that result in the following raw
p-values: $0.001, 0.015, 0.2, 0.8$. What are adjusted p-values using the
Bonferroni correction? How many tests are significant at the 0.05 level
after the correction?\footnote{
  If the adjusted value is greater than 1, we usually set it
  to 1 as the defintion of a p-value is given as a probability
  at we can safely always use 1 as an upper bound.
}

% SOLUTION

% p.adjust(c(0.001, 0.015, 0.2, 0.8), method = "bonferroni")

We simply multiple each of the p-values by $4$ to get: 
$0.004 0.060 0.800 1.000$. Only one test is significant.

%%%%%%%%%%%%%%%%%%%%%%%%%%%%%%%%%%%%%%%%%%%%%%%%%%%%%%%%%%%%%%%%%%%%%%%%%%%%%%%%%
% QUESTION

The FWER only considers whether we have made at least one false rejection
of one of the hypothesis tests. It does not matter, at least for FWER, if
we make one mistake or many. Thinking about this for a moment, we might
consider setting all of the p-values equal to the smallest p-value multiplied
by the number of tests. Why? Let's consider just two tests for hypotheses
$H_1$ and $H_2$ with p-values $p_1$ and $p_2$. Consider adjusting each 
of the p-values to be $2 \times \min(p_1, p_2)$. (a) If both $H_1$ and $H_2$
are true, what is the (upper bound) on the FWER if we use the adjusted
p-values at a level $\alpha$? (b) If neither $H_1$ and $H_2$ are true,
what is the FWER? (c) What can we say about the FWER if $H_1$ is true
but not $H_2$ (or vice-versa)?

% SOLUTION

(a) If both $H_1$ and $H_2$ are true, the probability that at least one
of the respective (uncorrected) p-values is less than $\alpha / 2$ is
$\alpha$, just as we have already seen. So, the FWER with the new corrections
is the same. In other words, we have a higher chance of making two errors
rather than just one, but the same chance of making at least one error.

(b) If both $H_1$ and $H_2$ are false, the FWER is zero. There is no
chance of making a false rejection of a null hypothesis because there
are no true null hypotheses.

(c) We cannot say anything about this latter case because we know nothing
about the distribution of $p_1$. We only have knowledge of the p-values 
under the null being correct. 

%%%%%%%%%%%%%%%%%%%%%%%%%%%%%%%%%%%%%%%%%%%%%%%%%%%%%%%%%%%%%%%%%%%%%%%%%%%%%%%%%
% QUESTION

Consider an adjusted version of the above procedure. We adjust the smallest 
p-value to be twice its original value, but set the larger p-value to be
the smaller of the adjusted smaller value or the original second largest
p-value (it sounds more complex than it is). (a) If both $H_1$ and $H_2$
are true, what is the (upper bound) on the FWER if we use the adjusted
p-values at a level $\alpha$? (b) If neither $H_1$ and $H_2$ are true,
what is the FWER? (c) What can we say about the FWER if $H_1$ is true
but not $H_2$ (or vice-versa)?

% SOLUTION

(a) This is still $\alpha$: all of the hypotheses are true, so making a
Type I error is driven by the smallest p-value.

(b) This is still $0$. There is no possibility of making a Type I error
as none of the null-hypotheses are true.

(c) This is where things get interesting. The FWER depends only on whether
the adjusted $p_1$ is less than $\alpha$. But, the adjustment can never 
make a p-value increase, and we know by definition that $p_1$ will only
be less than $\alpha$ with probability $\alpha$. So, we have the correct
FWER for all cases.


%%%%%%%%%%%%%%%%%%%%%%%%%%%%%%%%%%%%%%%%%%%%%%%%%%%%%%%%%%%%%%%%%%%%%%%%%%%%%%%%%
% QUESTION

The \textbf{Holm-Bonferroni} correction is a generalization of the above
for any number of tests. Consider a set of $m$ sorted p-values from smallest
to largest: $p_1, \ldots, p_m$. We adjust them according to the following
iterative procedure (setting $p'_0 = 0$):
\begin{align*}
p'_j &= \max\{ p'_{j-1}, \frac{p_j}{m + 1 - j} \}
\end{align*}
This will control the FWER at the level $\alpha$ for the same logic that we
derived above in the two-test case. What are the Holm-Bonferroni corrected
p-values from the first question? How many tests are significant at the 0.05
level after the correction?

% SOLUTION

The adjusted values are $0.004, 0.045, 0.400, 0.800$. We now have two tests
that are significant at the 0.05 level.
