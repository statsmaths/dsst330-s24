\documentclass{tufte-handout}

\usepackage{amssymb,amsmath}
% \usepackage{mathspec}
\usepackage{graphicx,grffile}
\usepackage{longtable}
\usepackage{booktabs}
\usepackage{mathtools}

\newtheorem{mydef}{Definition}
\newtheorem{thm}{Theorem}

\DeclareMathOperator*{\argmin}{arg\,min}
\DeclareMathOperator*{\argmax}{arg\,max}
\newcommand{\Lim}[1]{\raisebox{0.5ex}{\scalebox{0.8}{$\displaystyle \lim_{#1}\;$}}}
\newcommand{\E}{\mathbb{E}}
\newcommand{\Prob}{\mathbb{P}}
\newcommand{\V}{\text{Var}}
\newcommand{\iid}{\stackrel{iid}{\sim}}
\newcommand{\cblack}{\color{Black}}
\newcommand{\cblue}{\color{MidnightBlue}}

\providecommand{\tightlist}{%
  \setlength{\itemsep}{0pt}\setlength{\parskip}{0pt}}

\setlength{\parindent}{0em}
\setlength{\parskip}{12pt}

\begin{document}

\justify

{\LARGE Worksheet 07 (Solutions)}

\vspace*{18pt}


\textbf{1}. Consider a sample of size $5$ with the following values: $0$, $1$, $5$, $7$, $12$.
What are the sample mean and sample variance $\bar{X}$ and $S^2_X$?

% x <- c(0, 1, 5, 7, 12)
% var(x)

\textit{Solution:} The sample mean is:
\begin{align*}
\bar{X} &= \frac{0 + 1 + 5 + 7 + 12}{5} = 5
\end{align*}
And the sample variance is:
\begin{align*}
S^2_X &= \frac{(0-5)^2 + (1-5)^2 + (5-5)^2 + (7-5)^2 + (12-5)^2}{5 - 1} &= 23.5
\end{align*}

\textbf{2}. Consider collecting data from two populations. We collect $n=25$ observations
from the first group, with sample mean $7$ and sample variance $9$. From the 
second group, we have $m=30$ samples and a mean of $4$ with a sample variance
of $4$. What is the pooled variance $S^2_p$?

\textit{Solution:} The pooled variance is just the weighted averages of the sample variances.
That is:
\begin{align*}
S_p^2 &= \frac{(n-1) S_X^2 + (m-1) S_Y^2}{n + m - 2} \\
&= \frac{(24) \cdot 9 + (29) \cdot 4}{53} = 6.264 \\
\end{align*}

\textbf{3}. Using the data from above, construct a $99$\% confidence interval for the 
difference in the means. You can use the fact that $t_{0.01/2}(53) = 2.671$. 

% qt(1 - 0.01/2, df = 53)

\textit{Solution:} Using the formula from the table, we have:
\begin{align*}
(\bar{X} - \bar{Y}) \pm t_{\alpha/2} \cdot \sqrt{S_p^2 \times \left[\frac{1}{n} + \frac{1}{m}\right]} \\
(7-4) \pm 2.671 \cdot \sqrt{6.264 \times \left[\frac{1}{25} + \frac{1}{30}\right]} \\
(3) \pm 1.8103 \\
\end{align*}
Which we could also write as: $[1.19, 4.81]$.

\textbf{4}. Using the data from above, run a hypothesis test to see if the samples come from
distributions with the same variance using a $99$\% confidence level. Use the fact
that $f_{0.01/2}(24, 29) = 2.76$ and $f_{1-0.01/2}(24, 29) = 0.347$. 

% qf(1 - 0.01/2, df1 = 24, df2 = 29)
% qf(0.01/2, df1 = 24, df2 = 29)

\textit{Solution:} The test statistic for the ratio of the variances is $\frac{S_X^2}{S_Y^2} \cdot \Delta_0$
where $\Delta_0$ is the ratio of the variances. Here $\Delta_0 = 0$, so the test statistic
is $F = \frac{9}{4}=2.25$. The rejection region is:
\begin{align*}
\left\{F \leq f_{0.01/2} \right\} \cup \left\{F \leq f_{0.01/2} \right\} \\
\left\{F \leq 0.347 \right\} \cup \left\{F \leq 2.76 \right\} \\
\end{align*}
Our $F$ is equal to $2.25$, so neither in the left part nor the right part of the 
rejection region. Therefore, we \textbf{fail to reject the null hypothesis}.


\end{document}
