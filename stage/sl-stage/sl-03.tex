\documentclass{tufte-handout}

\usepackage{amssymb,amsmath}
% \usepackage{mathspec}
\usepackage{graphicx,grffile}
\usepackage{longtable}
\usepackage{booktabs}
\usepackage{mathtools}

\newtheorem{mydef}{Definition}
\newtheorem{thm}{Theorem}

\DeclareMathOperator*{\argmin}{arg\,min}
\DeclareMathOperator*{\argmax}{arg\,max}
\newcommand{\Lim}[1]{\raisebox{0.5ex}{\scalebox{0.8}{$\displaystyle \lim_{#1}\;$}}}
\newcommand{\E}{\mathbb{E}}
\newcommand{\Prob}{\mathbb{P}}
\newcommand{\V}{\text{Var}}
\newcommand{\iid}{\stackrel{iid}{\sim}}
\newcommand{\cblack}{\color{Black}}
\newcommand{\cblue}{\color{MidnightBlue}}

\providecommand{\tightlist}{%
  \setlength{\itemsep}{0pt}\setlength{\parskip}{0pt}}

\setlength{\parindent}{0em}
\setlength{\parskip}{12pt}

\begin{document}

\justify

{\LARGE Worksheet 03 (Solutions)}

\vspace*{18pt}


\textbf{1}. Let $X_1, \ldots, X_n \sim N(\mu_X, \sigma_X^2)$ be a random sample. Using
the results from the handout, construct a pivot statistic $T$ as a function
of $\bar{X}$, $S_X^2$, $\mu_X$, and $\sigma_X^2$ that has a distribution of
$t(n-1)$. Do not simplify.

\textit{Solution:} We have the following, by simply plugging in the results from the handout:
\begin{align*}
T &= \frac{\frac{\mu_X - \bar{X}}{\sqrt{\sigma_X^2 / n}}}{\sqrt{\frac{(n-1)S^2_X}{\sigma_X^2 \cdot (n-1)}}}
\end{align*}

\textbf{2}. Simplify the form of the $T$ statistic. It should no longer have any
$\sigma_X^2$ terms (in fact this is the whole point of this specific form). 
Try to write the solution with $(\mu - \bar{X})$ in the numerator and
everything else in the denominator.

\textit{Solution:} Simplifying, we see that:
\begin{align*}
T &= \frac{\frac{\mu_X - \bar{X}}{\sqrt{\sigma_X^2 / n}}}{\sqrt{\frac{(n-1)S^2_X}{\sigma_X^2 \cdot (n-1)}}} \\
&= \frac{\mu_X - \bar{X}}{\sqrt{S^2_X / n}}
\end{align*}
As desired.

\textbf{3}. Let $t_\alpha(k)$ be the tail probability of a T-distribution with $k$
degrees of freedom, just as we had with $z_\alpha$ on the handout. Following
the same procedure with the example on the handout, construct a confidence
interval with confidence level $(1 - \alpha)$ for $\mu_X$. Write the solution
as $\bar{X} \pm \Delta$ for some $\Delta$.

\textit{Solution:} Starting with the pivot statistic, we have:
\begin{align*}
\Prob\left[ -t_\alpha(n-1) \leq T \leq t_\alpha(n-1) \right] &= 1 - \alpha \\
\Prob\left[ -t_\alpha(n-1) \leq \frac{\mu_X - \bar{X}}{\sqrt{S^2_X / n}} \leq t_\alpha(n-1) \right] &= 1 - \alpha \\
\Prob\left[ -t_\alpha(n-1) \cdot \sqrt{S^2_X / n} \leq (\mu_X - \bar{X}) \leq t_\alpha(n-1) \cdot \sqrt{S^2_X / n} \right] &= 1 - \alpha \\
\Prob\left[ \bar{X} - t_\alpha(n-1) \cdot \sqrt{S^2_X / n} \leq \mu_X \leq \bar{X} + t_\alpha(n-1) \cdot \sqrt{S^2_X / n} \right] &= 1 - \alpha \\
\end{align*}
Which we can write as the following:
\begin{align*}
\bar{X} &\pm t_\alpha(n-1) \cdot \sqrt{\frac{S^2_X}{n}}
\end{align*}

\textbf{4}. We will go back to the story about the fish. Say we sample $25$ fish and
have a sample mean of $12.1$ centimeters and a sample variance of $6$
centimeters squared. Given that $t_{0.01/2}(24)$ is approximately equal
to $2.797$, derive the confidence interval for the mean.

% 12.1 + c(-1, 1) * 2.797 * sqrt(6/25)

\textit{Solution:} We have:
\begin{align*}
12.1 &\pm \left[ 2.797 \cdot \sqrt{\frac{6}{25}} \right] \\
12.1 &\pm 1.37
\end{align*}
Or, we can also write this as $[10.73, 13.47]$.

\textbf{5}. Now, let's build a confidence interval for the variance. The chi-squared 
distribution is not symmetric, so we need to start with a more general
form of the equation with the pivot statistic (the last equation on the
handout). Namely, we have:
\begin{align*}
\mathbb{P}\left[ \chi^2_{1 - \alpha/2}(n-1) \leq \frac{(n-1)S^2_X}{\sigma_X^2} \leq \chi^2_{\alpha/2}(n-1) \right]
\end{align*}
To manipulate this into a confidence interval, first take the (multiplicative)
inverse of all three terms. Note that for positive numbers, taking the inverse
of both sides of an inequality reverses the sign of the inequality. Then,
simplify to get a confidence interval of $\sigma_X^2$.

\textit{Solution:} Starting with the pivot statistic, we have:
\begin{align*}
\mathbb{P}\left[ \frac{1}{\chi^2_{1 - \alpha/2}(n-1)} \geq \frac{\sigma_X^2}{(n-1)S^2_X} \geq \frac{1}{\chi^2_{\alpha/2}(n-1)} \right] \\
\mathbb{P}\left[ \frac{(n-1)S^2_X}{\chi^2_{1 - \alpha/2}(n-1)} \geq \sigma_X^2 \geq \frac{(n-1)S^2_X}{\chi^2_{\alpha/2}(n-1)} \right] \\
\end{align*}
And we now have a confidence interval for the variance. Not too bad, right?
Note that we cannot cancel the $(n-1)$ terms because while the top is the number
$n-1$, the bottom is a function argument. As an analogy, this would be similar 
to doing this: $\frac{x}{\sin(x)} = \frac{1}{\sin}$, which is clearly nonsensical.

\textbf{6}. Given that $\chi^2_{0.01/2}(24) \approx 45.56$ and $\chi^2_{1-0.01/2}(24) \approx 9.88$,
what is the 99\% confidence interval for the variance of the lengths of the
fish from our example data?

% qchisq(1 - 0.01/2, df = 24)
% 24 * 6 / 45.56
% 24 * 6 / 9.88

\textit{Solution:} The lower bound is:
\begin{align*}
\frac{(n-1)S^2_X}{\chi_{\alpha/2}(n-1)} &= \frac{24 \cdot 6}{9.88} = 3.16
\end{align*}
And the upper bound is:
\begin{align*}
\frac{(n-1)S^2_X}{\chi_{\alpha/2}(n-1)} &= \frac{24 \cdot 6}{45.56} = 14.57
\end{align*}
So the confidence interval is $[3.16, 14.57]$.


\end{document}
