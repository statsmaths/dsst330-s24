\documentclass{tufte-handout}

\usepackage{amssymb,amsmath}
% \usepackage{mathspec}
\usepackage{graphicx,grffile}
\usepackage{longtable}
\usepackage{booktabs}
\usepackage{mathtools}

\newtheorem{mydef}{Definition}
\newtheorem{thm}{Theorem}

\DeclareMathOperator*{\argmin}{arg\,min}
\DeclareMathOperator*{\argmax}{arg\,max}
\newcommand{\Lim}[1]{\raisebox{0.5ex}{\scalebox{0.8}{$\displaystyle \lim_{#1}\;$}}}
\newcommand{\E}{\mathbb{E}}
\newcommand{\Prob}{\mathbb{P}}
\newcommand{\V}{\text{Var}}
\newcommand{\iid}{\stackrel{iid}{\sim}}
\newcommand{\cblack}{\color{Black}}
\newcommand{\cblue}{\color{MidnightBlue}}

\providecommand{\tightlist}{%
  \setlength{\itemsep}{0pt}\setlength{\parskip}{0pt}}

\setlength{\parindent}{0em}
\setlength{\parskip}{12pt}

\begin{document}

\justify

{\LARGE Worksheet 20 (Solutions)}

\vspace*{18pt}


\textbf{1}. Find the Jeffreys prior for estimating the estimating the mean of a normal
distribution with a known variance $\sigma^2$. You can assume we have only one
observation $X$. What is the corresponding
Bayesian point estimator and how does it compare to the MLE?

\textit{Solution:} We have, from the results last time, the following:
\begin{align*}
\sqrt{\mathcal{I}(p)} &= \sqrt{\sigma^{-2}} \propto 1.
\end{align*}
So, this is the improper distribution of a uniform distribution over the entire
real line. Therefore, the Bayesian estimator is just equal to the MLE.

\textbf{2}. Find the Jeffreys prior for estimating the estimating the parameter $p$ from a
Binomial with a known value $n$. What is the corresponding Bayesian point estimator?
What does this mean in the case when $n=1$ and $X=0$ and in the case when $n=1$ and $X=1$?

\textit{Solution:} We have, from the results last time, the following:
\begin{align*}
\sqrt{\mathcal{I}(p)} &= \sqrt{p(1-p)} = p^{1/2} (1 - p)^{1/2} = p^{1-1/2} (1-p)^{1-1/2}.
\end{align*}
This is proportional to a $Beta(1/2,1/2)$. Therefore, from our previous notes, the 
point estimator is:
\begin{align*}
\widehat{p}_{Bayes} &= \frac{X + 1/2}{1/2 + 1/2 + n} = \frac{X + 1/2}{1 + n}.
\end{align*}
So, when $n=1$ and $X=0$ we guess $p=0.25$, and when $X=1$ we guess $p=0.75$.

\textbf{3}. Find the Jeffreys prior for estimating the estimating the parameter $\lambda$ from
a Poisson. Write down a formula that gives, up to a constant,
the posterior distribution. Note that you will not be able to relate this to a 
known distribution on our chart.

\textit{Solution:} We have, from the results last time, the following:
\begin{align*}
\sqrt{\mathcal{I}(\lambda)} &= \sqrt{\lambda^{-1}} = \lambda^{-1/2}.
\end{align*}
This implies that the posterior distribution is:
\begin{align*}
f(\lambda | x) &\propto \frac{\lambda^{x-1/2} e^{-\lambda}}{x!}
\end{align*}
This should be a proper distribution, but I am not aware of a way to analytically
find the normalizing constant. I believe that numerical techniques such as Gibbs
sampling are needed to find the Bayes point estimator.

\textbf{4}. The Fisher information for the geometric distribution is $\mathcal{I}(p) = \frac{(1-p)}{p^2}$.
Find the Jeffreys prior for estimating the estimating the parameter $p$ from
a geometric distribution. What is, more-or-less, this distribution?\footnote{
  It should line up with one of the results on the table, but the hyperparameter
  is out of bounds. That's okay though. It just means we have an improper prior.
  All of the results still hold. 
} What is the corresponding Bayesian point estimator? Using previous results,
you should be able to do this for a sample of size $n$.

\textit{Solution:} We have, from the results last time, the following:
\begin{align*}
\sqrt{\mathcal{I}(p)} &= \sqrt{\frac{(1-p)}{p^2}} = p^{-1} (1 - p)^{-1/2}.
\end{align*}
This is mathematically equivalent to the distribution of a $Beta(0, 1/2)$,
however this is not a proper distribution when $\alpha = 0$. Not a problem
though! We can still use our formula for the posterior that 
$p | X \sim Beta(a + n, b + \sum_i x_i - n)$. In this case, we have
$p | X \sim Beta(n, 1/2 + \sum_i x_i - n)$, which gives a Bayesian point
estimator of:
\begin{align*}
\widehat{p} &= \frac{n}{1/2 + \sum_i x_i}.
\end{align*}


\end{document}
