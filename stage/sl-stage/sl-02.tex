\documentclass{tufte-handout}

\usepackage{amssymb,amsmath}
% \usepackage{mathspec}
\usepackage{graphicx,grffile}
\usepackage{longtable}
\usepackage{booktabs}
\usepackage{mathtools}

\newtheorem{mydef}{Definition}
\newtheorem{thm}{Theorem}

\DeclareMathOperator*{\argmin}{arg\,min}
\DeclareMathOperator*{\argmax}{arg\,max}
\newcommand{\Lim}[1]{\raisebox{0.5ex}{\scalebox{0.8}{$\displaystyle \lim_{#1}\;$}}}
\newcommand{\E}{\mathbb{E}}
\newcommand{\Prob}{\mathbb{P}}
\newcommand{\V}{\text{Var}}
\newcommand{\iid}{\stackrel{iid}{\sim}}
\newcommand{\cblack}{\color{Black}}
\newcommand{\cblue}{\color{MidnightBlue}}

\providecommand{\tightlist}{%
  \setlength{\itemsep}{0pt}\setlength{\parskip}{0pt}}

\setlength{\parindent}{0em}
\setlength{\parskip}{12pt}

\begin{document}

\justify

{\LARGE Worksheet 02 (Solutions)}

\vspace*{18pt}


\textbf{1}. Assume that $\mathcal{G}$ is a normal distribution. What is the distribution
of the following quantity from the sample mean $\bar{X}$ from a random sample
with $n$ observations?
\begin{align*}
\left[\frac{\bar{X} - \mu_X}{\sigma_X / \sqrt{n}} \right]^2.
\end{align*}
This should be a short answer based on what you derived last time.

\textit{Solution:} The random variable inside of the square is a standard normal, so
the squared quantity has a $\chi^2(1)$ distribution.

\textbf{2}. Now, let's consider the following quantity, which we will temporarily
give a name of $Y$ (it's not a quantity we need often, so there is not
a standard symbol for it):
\begin{align*}
Y &= \frac{1}{\sigma_X^2} \times \sum_i \left[ X_i - \mu_X \right]^2 
\end{align*}
What is the distribution of $Y$?

\textit{Solution:} Moving the constant inside, we see that:
\begin{align*}
Y &=  \sum_i \left[ \frac{X_i - \mu_X}{\sigma_X} \right]^2 
\end{align*}
Each of the components of the sum have a distribution of $N(0, 1)$ and
every component is independent. So, $Y \sim \chi^2(n)$.

\textbf{3}. The quantity $Y$ looks similar to $S_X^2$. We will use a common trick to
get $Y$ in terms of $S_X^2$: adding and subtracting the quantity $\bar{X}$
inside of the terms inside the sum. We can put the constant factor in later,
and so let's start with the following equality:
\begin{align*}
\sum_i \left[ X_i - \mu_X \right]^2 &= \sum_i \left[ X_i - \bar{X} + \bar{X} - \mu_X \right]^2  \\
&= \sum_i \left[ (X_i - \bar{X}) + (\bar{X} - \mu_X) \right]^2 
\end{align*}
Make sure that you see why this is valid! Starting with the
formula above, distribute the square. You should have three different summation
terms. Simplify by showing that the cross-term (the one with the $2$ in it) is
zero and another one of the terms is a constant in terms of the index $i$. The
third term should look similar to $S_X^2$. This is somewhat tricky. Make sure
you check the anwer before moving on.

\textit{Solution:} We start by the straightforward
distribution of the squared term:
\begin{align*}
\sum_i \left[ X_i - \mu_X \right]^2 &= \sum_i \left[ (X_i - \bar{X}) + (\bar{X} - \mu_X) \right]^2 \\
&= \sum_i \left[ (X_i - \bar{X})^2 + (\bar{X} - \mu_X)^2 + 2 (X_i - \bar{X}) \cdot (\bar{X} - \mu_X) \right] \\
&= \sum_i (X_i - \bar{X})^2 + \sum_i (\bar{X} - \mu_X)^2 + \sum_i 2 (X_i - \bar{X}) \cdot (\bar{X} - \mu_X)
\end{align*}
Terms that do not have an $i$ index can come outside of the summation. The middle
term is all constant, so we just remove the sum by multiplying by $n$ (the number
of terms in the series):
\begin{align*}
\sum_i \left[ X_i - \mu_X \right]^2 
&= \sum_i (X_i - \bar{X})^2 + n \cdot (\bar{X} - \mu_X)^2 + 2 \cdot (\bar{X} - \mu_X) \cdot \sum_i  (X_i - \bar{X}) \\
&= \sum_i (X_i - \bar{X})^2 + n \cdot (\bar{X} - \mu_X)^2
\end{align*}
The last step comes from the fact that $\sum_i (X_i - \bar{X})$ must be zero. If
you do not believe that, distribute the summation and work out the details to see
why. 

\textbf{4}. Divide both sides of your previous answer by $\sigma_X^2$. You
should have one term on the left and two on the right. Make one of the terms
on the right look like quantity in question 1.

\textit{Solution:} \begin{align*}
\sum_i \left[ \frac{X_i - \mu_X}{\sigma_X} \right]^2 
&= \sum_i (X_i - \bar{X})^2 + n \cdot (\bar{X} - \mu_X)^2 + 2 \cdot (\bar{X} - \mu_X) \cdot \sum_i  (X_i - \bar{X}) \\
&= \frac{1}{\sigma_X^2} \sum_i (X_i - \bar{X})^2 + \frac{n}{\sigma_X^2} \cdot (\bar{X} - \mu_X)^2 \\
&= \frac{1}{\sigma_X^2} \sum_i (X_i - \bar{X})^2 + \left[\frac{\bar{X} - \mu_X}{\sigma_X / \sqrt{n}} \right]^2.
\end{align*}

\textbf{5}. Using the previous set of results, what is the distribution of
the following quantity?
\begin{align*}
\frac{1}{\sigma_X^2} \sum_i (X_i - \bar{X})^2 &= \frac{(n-1) S_X^2}{\sigma_X^2}
\end{align*}

\textit{Solution:} The left-hand side of the previous answer is $\chi^2(n)$ and the right-hand
side is the sum of two independent terms: a $\chi^2(1)$ and the value above. Therefore,
the term above must be a $\chi^2(n-1)$ (because then there sum would be a $\chi^2(n)$,
as required). 

\textbf{6}. From probability theory, we have that the expected value of a random variable with
a chi-squared distribution with $k$ degrees of freedom is $k$. Its variance is $2k$.
Take the expected value of the quantity from the previous question and
simplify to get the expected value of $S_X^2$. You should see that $S_X^2$ is
an unbiased estimator of $\sigma_X^2$.

\textit{Solution:} We have:
\begin{align*}
\E \left[ \frac{(n-1) S_X^2}{\sigma_X^2} \right] &= n-1 \\
\frac{(n-1)}{\sigma_X^2} \cdot \E \left[ S_X^2 \right] &= n-1 \\
\E \left[ S_X^2 \right] &= \sigma_X^2
\end{align*}
So the expected value of the sample variance is equal to the population variance.
We say that this is an unbiased estimator of the variance, a concept that we will
return to in the next unit.

\textbf{7}. Take the
variance of the quantity you started with in the previous question and simplify
to get the variance of $S_X^2$.

\textit{Solution:} This is relatively straightforward based on the chi-squared distribution:
\begin{align*}
\V \left[ \frac{(n-1) S_X^2}{\sigma_X^2} \right] &= 2(n-1) \\
\frac{(n-1)^2}{\sigma_X^4} \cdot \V \left[ S_X^2 \right] &= 2(n-1) \\
\V \left[ S_X^2 \right] &= \frac{\sigma_X^4}{n-1}
\end{align*}


\end{document}
