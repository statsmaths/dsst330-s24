\documentclass{tufte-handout}

\usepackage{amssymb,amsmath}
% \usepackage{mathspec}
\usepackage{graphicx,grffile}
\usepackage{longtable}
\usepackage{booktabs}
\usepackage{mathtools}

\newtheorem{mydef}{Definition}
\newtheorem{thm}{Theorem}

\DeclareMathOperator*{\argmin}{arg\,min}
\DeclareMathOperator*{\argmax}{arg\,max}
\newcommand{\Lim}[1]{\raisebox{0.5ex}{\scalebox{0.8}{$\displaystyle \lim_{#1}\;$}}}
\newcommand{\E}{\mathbb{E}}
\newcommand{\Prob}{\mathbb{P}}
\newcommand{\V}{\text{Var}}
\newcommand{\iid}{\stackrel{iid}{\sim}}
\newcommand{\cblack}{\color{Black}}
\newcommand{\cblue}{\color{MidnightBlue}}

\providecommand{\tightlist}{%
  \setlength{\itemsep}{0pt}\setlength{\parskip}{0pt}}

\setlength{\parindent}{0em}
\setlength{\parskip}{12pt}

\begin{document}

\justify

{\LARGE Worksheet 14 (Solutions)}

\vspace*{18pt}


\textbf{1}. Consider a simple linear regression where we know that $b_0 = 0$. You can write
$b_1 \rightarrow b$ to simplify the notation. Write down the likelihood function
for the sample. Do not yet simplify.

\textit{Solution:} The likelihood is given by:
\begin{align*}
\mathcal{L}(b; y_1, \ldots, y_n) &= 
\prod_i \frac{1}{(2 \pi \sigma^2)^{1/2}} \times e^{-\frac{1}{2\sigma^2}(y_i - x_i b)^2}.
\end{align*}

\textbf{2}. Now, (a) compute the log-likelihood function and simplify. (b) Without doing any
calculus (that is, just looking at the function), maximizing the log-likelihood
with respect to $b$ is equivalent to minimizing what quantity in terms of $y_i$,
$x_i$, and $b$? (c) Why might it make sense to minimize this quantity?
Note: Ask me about the correct solution before proceeding. 

\textit{Solution:} (a) Taking the logarithm and simplifying yields:
\begin{align*}
\mathcal{l}(b; y_1, \ldots, y_n) &= 
\sum_i \log\left(\frac{1}{(2 \pi \sigma^2)^{1/2}}\right) - \frac{1}{2\sigma^2}(y_i - x_i b)^2 \\
&=  -\frac{n}{2}\log(2 \pi \sigma^2) - \frac{1}{2\sigma^2} \sum_i (y_i - x_i b)^2.
\end{align*}
(b) Looking at this, we see that to maximize the log-likelihood with $b$, we need
to minimize the quantity $\sum_i (y_i - x_i b)^2$. (c) This is actually a logical
thing to do, because these are the squared sums of the residuals, the amount that
we are missing the $y_i$'s by our regression line. Making these as small as 
possible is a reasonable thing; it is also where the term \textit{best-fit line}
for the solution comes from.

\textbf{3}. Take the derivative of the quantity that you had in part (b) from the previous
question with respect to the parameter $b$. Set this equal to zero to get the 
MLE.

\textit{Solution:} The derivative of the sum of squares is:
\begin{align*}
\frac{\partial}{\partial b} \sum_i (y_i - x_i b)^2 &= -2 \sum_i x_i \cdot (y_i - x_i b)
\end{align*}
And solving for zero gives:
\begin{align*}
-2 \sum_i x_i \cdot (y_i - x_i b) &= 0\\
\sum_i x_i \cdot y_i - x_i &= \sum_i x_i^2 \widehat{b} \\
\frac{\sum_i x_i \cdot y_i}{\sum_i x_i^2} &= \widehat{b}.
\end{align*}

\textbf{4}. What obsevations will have the most influence on the estimate of the slope?
Does this make sense?

\textit{Solution:} Observations farther from the origin will have a higher impact on the output.
This makes sense because we are measuring the slope of a line through the origin.
Since the variance of $Y_i$ is fixed, we have more signal in points that are 
farther from the origin.

\textbf{5}. What is the distribution of the MLE of $b$? Is the estimator unbiased? Under
what conditions will it be consistent? Note: This will take several steps.

\textit{Solution:} We see quickly that $\widehat{b}$ is a sum of independent normals, so it will 
have a normal distribution. We need only to figure out its mean and variance.
These are given by:
\begin{align*}
\mathbb{E} \widehat{b} &= \mathbb{E} \left( \frac{\sum_i x_i \cdot y_i}{\sum_i x_i^2} \right) \\
&= \left( \frac{\sum_i x_i \cdot \mathbb{E} y_i}{\sum_i x_i^2} \right) \\
&= \left( \frac{\sum_i x_i \cdot x_i b}{\sum_i x_i^2} \right) \\
&= \left( \frac{\sum_i x_i^2 b}{\sum_i x_i^2} \right) \\
&= b \cdot \left( \frac{\sum_i x_i^2}{\sum_i x_i^2} \right) \\
&= b
\end{align*}
So, we see that it is unbiased. The variance is given by:
\begin{align*}
Var (\widehat{b}) &= Var \left( \frac{\sum_i x_i \cdot y_i}{\sum_i x_i^2} \right) \\
&= \left( \frac{\sum_i x_i^2 \cdot Var (y_i)}{\left(\sum_i x_i^2\right)^2} \right) \\
&= \left( \frac{\sum_i x_i^2 \cdot \sigma^2}{\left(\sum_i x_i^2\right)^2} \right) \\
&= \sigma^2 \left( \frac{\sum_i x_i^2}{\left(\sum_i x_i^2\right)^2} \right) \\
&= \sigma^2 \cdot  \frac{1}{\sum_i x_i^2} = \frac{\sigma^2}{\sum_i x_i^2}.
\end{align*}
The variance will limit to zero as long as $\sum_i x_i^2 \rightarrow \infty$,
generally the case as long as we have data points $x_i$ that are not limiting
to the origin in some strange way.

\textbf{6}. Go back to the full log-likelihood function. Take the derivative with respect
to $\sigma^2$ (remember, this is a single parameter, not the square of a parameter).
Set this to zero and solve to get the MLE of $\sigma^2$. Does this equation make
sense to you?

\textit{Solution:} I will set $v = \sigma^2$ for clarify. Then, we have, at the optimal point of
$b = \widehat{b}$, the following:
\begin{align*}
\frac{\partial}{\partial v} (\cdot) &= \frac{-n}{2} \frac{1}{2\pi v} \cdot (2\pi) + \frac{1}{2v^2} \sum_i \widehat{y}_i^2 \\
&= \frac{-n}{2v} + \frac{1}{2v^2} \sum_i \widehat{y}_i^2
\end{align*}
Setting this to zero yields:
\begin{align*}
\frac{n}{2\widehat{v}} &= \frac{1}{2\widehat{v}^2} \sum_i \widehat{y}_i^2 \\
\frac{2\widehat{v}^2}{2\widehat{v}} &= \frac{1}{n} \sum_i \widehat{y}_i^2 \\
\widehat{v} &= \frac{1}{n} \sum_i \widehat{y}_i^2.
\end{align*}
This should make sense because it measures the squared size of the residuals,
which we expect to be normally distributed with variance $\sigma^2$.

\textbf{7}. The MLE estimator for $\sigma^2$ is biased, but we can fix this by dividing by
$n-1$ instead of $n$, just as we did with the one-sample mean. This unbiased
version is independent of $\widehat{b}$. If we take this unbiased estimator and
divide by $\sigma^2$, we will have a chi-squared distribution with $n-1$ degrees
of freedom. Using this, create a pivot statistic that depends only on $b$ and not
$\sigma^2$.

\textit{Solution:} This is just a matter of plugging in our answers to the previous questions and
using the formula for a T-statistic:
\begin{align*}
T &= \frac{\frac{\widehat{b} - b}{\sqrt{\sigma^2 / \sum_i x_i^2}}}{
  \sqrt{ \frac{n-2}{n-2} \cdot \frac{1}{\sigma^2} \sum_i \widehat{y}_i^2 }
} \\
&=
\frac{\widehat{b} - b}{\sqrt{\frac{\sum_i \widehat{y}_i^2}{\sum_i x_i^2}}}.
\end{align*}


\end{document}
